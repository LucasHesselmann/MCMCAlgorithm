\documentclass[10pt, a4paper, leqno, twoside, bibliography=totocnumbered, final]{scrartcl}
%Wasserzeichen
%\usepackage{draftwatermark}
%\SetWatermarkText{BIO-HAZARD}
%\SetWatermarkLightness{0.5}
%\SetWatermarkScale{0.8}
%europäischer Zeichensatz/Silbentrennung
\usepackage[T1]{fontenc}
\usepackage{lmodern}
%Eingabe von Sonderzeichen
\usepackage[utf8]{inputenc}
%deutsche Sprache
\usepackage[ngerman]{babel}
%schönerer Textsatz
\usepackage{microtype}
\usepackage{ellipsis}
%für mathematische Formeln
\usepackage{latexsym}
\usepackage{amsfonts}
%Bachelortitelseite
\usepackage{BA_Titelseite}
%für die Aufzählungszeichen
\usepackage{paralist}
%für Gleichungen
\usepackage{amsmath}
%für newtheorem
\usepackage{amsthm}
%für Skriptbuchstaben
\usepackage{mathrsfs}
%header
\usepackage{fancyhdr}
\fancypagestyle{plain}{%
\fancyhf{} % clear all header and footer fields
%\fancyfoot[LE,RO]{\thepage}
%\fancyhead[CE]{\textsc{Reflektierte Brownsche Bewegung}}
\fancyhead[CE]{\begin{small} \leftmark \end{small}}
\fancyhead[CO]{\begin{small} \rightmark \end{small}}
\fancyfoot[LE]{\thepage}
\fancyfoot[RO]{\thepage}
\renewcommand{\headrulewidth}{0.4pt}
\renewcommand{\footrulewidth}{0pt}}
\pagestyle{plain}
%Literaturverzeichnis
\usepackage[babel, german=quotes]{csquotes}
\usepackage{natbib}

%\setlength{\textheight}{560pt}






%Namen des Verfassers der Arbeit
\authornew{Lucas Frederic Marcus Hesselmann}

%Geburtsdatum des Verfassers
\geburtsdatum{10. August 1988}
%Gebortsort des Verfassers
\geburtsort{Osnabrück}
%Datum der Abgabe der Arbeit
\date{\today}

%Name des Betreuers
% z.B.: Prof. Dr. Peter Koepke
\betreuer{Betreuer: Prof. Dr. Karl-Theodor Sturm}
%Name des Instituts an dem der Betreuer der Arbeit ttig ist.
%z.B.: Mathematisches Institut
\institut{Institut f\"{u}r angewandte Mathematik}
%Titel der Bachelorarbeit
\title{Reflektierte Brownsche Bewegung}
%Do not change!
\ausarbeitungstyp{Bachelorarbeit Mathematik}

%Umgebungen für Theoreme, Sätze, Lemmata, Definitionen und Beispiele
%Nummern vorne
\swapnumbers
%Umgebungen durchnummeriert in jeder section
\theoremstyle{definition}
\newtheorem{defin}{Definition}[section]
\theoremstyle{plain}% default
\newtheorem{thm}[defin]{Theorem}
\newtheorem{satz}[defin]{Satz}
\newtheorem{lemma}[defin]{Lemma}
\newtheorem{kor}[defin]{Korollar}
\theoremstyle{remark}
\newtheorem*{bem}{Bemerkung}

\DeclareRobustCommand{\qvar}[1]{\ensuremath{\left\langle {#1} \right\rangle}}
%Nummerierung der Gleichungen
\renewcommand{\theequation}{\thesection.\arabic{equation}}

%\pagestyle{headings}

\begin{document}

\maketitle

\tableofcontents

\newpage

\section{Einleitung}

Zwischen der Wahrscheinlichkeitstheorie und der Analysis besteht eine besondere Verbindung, deren Studium bereits auf Andrej N. Kolmogorov und damit auf die Anfänge der modernen Wahrscheinlichkeitstheorie zurückgeht. Hervorzuheben in diesem Kontext ist der starke Zusammenhang der \emph{Brownschen Bewegung} mit partiellen Differentialgleichungen. Denn die Lösungen zu verschiedenen elliptischen und parabolischen partiellen Differentialgleichungen können als Erwartungswert von stochastischen Funktionalen ausgedrückt werden. Eins der einfachsten Beispiele für einen elliptischen Operator ist das \emph{Dirichlet-Randwertproblem}:
\begin{equation}
\begin{cases}
\Delta u  = 0 & \text{ auf $D$ }\\
 u  = f & \text{ auf $ \partial D $ }.
\end{cases}
\end{equation}

Mithilfe der \emph{absorbierenden Brownschen Bewegung} lässt sich das Dirichlet-Problem charakterisieren, und eine Lösung angeben:
\begin{equation}
u(x) = E^x \left[ f(X_{\tau_D})\right]
\end{equation}
mit $ \tau_D $ als Trefferzeit des Randes $ \partial D $ und $ X $ eine Brownsche Bewegung mit Startpunkt $ x \in D $.

Das Ziel dieser Arbeit ist es eine Lösung für das \emph{Neumann-Randwertproblem}
\begin{equation}
\begin{cases}
\Delta u = 0 & \text{ auf $D$ }\\
\frac{\partial}{\partial n} u = f & \text{ auf $ \partial D $ },
\end{cases}
\end{equation}
durch die probabilistische Methode anzugeben. Die komplexere Struktur der Neumann-Randbedigung im Vergleich zu Dirichlet-Randbedingungen durch die Festlegung der Normalenableitung auf dem Rand erfordert einen größeren Aufwand als nur das passende Stoppen der Brownschen Bewegung. Es gilt, die Brownsche Bewegung geeignet fortzusetzen. Dies geschieht durch Reflexion am Rand entlang der Normalen. Dadurch entsteht ein zusätzlicher stochastischer Prozess, die Lokalzeit der Brownschen Bewegung auf dem Rand. Allein für den Halbraum oder glatt berandete Gebiete ist diese Konstruktion nicht trivial  siehe Kapitel~\ref{sec:Pfadweise-Reflexion}, und für Lipschitz Gebiete wird sogar die Theorie der Dirichlet Formen verwendet, wie in Kapitel~\ref{sec:Reflektierte-Brownsche-bewegung-auf-Lipschitz-Gebieten} zu sehen. Mit der so konstruierten \emph{reflektierten Brownschen Bewegung} lässt sich die Lösung des Neumann-Randwertproblems angeben (siehe Abschnitt~\ref{sec:NRWP-section}):
\begin{equation}
u(x) = \lim_{t \to \infty} \frac{1}{2} E^x \left[ \int_0^t f(X_s) dL_s \right],
\end{equation}
wobei $ L $ die Lokalzeit der reflektierten Brownschen Bewegung $ X $ ist.




\newpage

\section{Pfadweise Reflexion}
\label{sec:Pfadweise-Reflexion}

Ziel ist es einen Lösungsansatz für stochastische Differentialgleichungen mit reflektierenden Randbedingungen für glatt berandete mehrdimensionale Gebiete basierend auf dem Skorokhod-Problem zu finden.
Dieses Problem ist von vielen Autoren für verschiedene Regularitäten des Gebiets untersucht worden. Besonders hervorzuheben, da sie als Hauptquelle dieses Kapitels dienen, sind die Arbeiten von H. Tanaka \cite{Tanaka}, der den Fall von konvexen Gebieten untersucht, von P.L. Lions und A.S. Sznitman \cite{Lions-Sznitman} und von R.F. Anderson und S. Orey \cite{Anderson-Orey}, deren Lokalisationstechnik hier für hinreichend glatte Gebiete verwendet wird. 

Um die Ergebnisse und Methoden in dieser Arbeit genauer zu erläutern, betrachte man das folgende Problem: Im Fall von normaler Reflexion an der Innenseite einer glatt berandeten und zusammenhängenden offenen Menge $D$ in $ \mathbb{R}^d $ wird gezeigt, dass für jedes $ w \in C([0, \infty]; \mathbb{R}^d) $ mit $ w(0) \in \overline{D} $ eine eindeutige Lösung $ (x,l) $ des \emph{Skorokhod-Problems (\ref{eq:S*})} existiert:
\begin{equation}
\label{eq:S*}
\left \{ \begin{aligned}
& x \in C([0, \infty); \overline{D}), \; l \in C([0, \infty); \mathbb{R}^d), \; |l|_T < \infty \; \text{für alle } T < \infty, \; \text{und} \\  
& l(t) = \int_0^t \mathsf{1}_{\{ x(s) \in \partial D \} }(s) n(x(s)) d|l|(s), \quad \text{so dass} \\
& x(t) = w(t) + l(t) \qquad \text{für alle $ t \geq 0 $}.
\end{aligned} \right .
\end{equation}
wobei $|l|(t)$ für die totale Variation von $l$ auf $ [0,t] $ steht und $n(a)$ die innere Normale an $ a \in \partial D $ bezeichnet.

Dieses deterministische Problem wird für den Halbraum $ \mathbb{R}_+ = \{x > 0 \} \subset \mathbb{R} $ untersucht und gelöst. Die Ergebnisse lassen sich auf den mehrdimensionalen Halbraum $ \mathbb{R}_+^d = \{ x \in \mathbb{R}^d : x_1 > 0 \} $ übertragen, indem man $ \mathbb{R}_+^d $ als $ \mathbb{R}_+ \times \mathbb{R}^{d-1} $ auffasst. Für die erste Komponente kann man das Skorokhod-Problem lösen und für die restlichen $(d-1)$-Koordinaten gibt es keine Einschränkungen. Durch die in Abschnitt~\ref{sec:lokalisation} angegebene Lokalisationstechnik lassen sich die Ergebnisse auf hinreichend glatt berandete Gebiete in $ \mathbb{R}^d $ übertragen, indem man das Problem auf dem Halbraum mittels lokalen Koordinaten transformiert.\newline

Nachdem man das deterministische Problem (\ref{eq:S*}) gelöst hat, kann man stochastische Differentialgleichungen mit Reflexion entlang der Normalen lösen:

Sei $ (\Omega, F, F_t , P) $ ein die üblichen Bedingungen erfüllender filtrierter Wahrscheinlichkeitsraum, mit $ B_t $ einer $F_t$-Brownschen Bewegung. Dann finde ein stetiges $F_t$-adaptiertes Semimartingal $X$, so dass gilt:
\begin{equation}
\label{eq:SDE*}
\left \{ \begin{aligned}
& X_{t} = x_{0} + \int_{0}^{t} \sigma (s, X_{s}) dB_{s} + \int_0^t b(s,X_{s})ds + L_{t}, \\ 
& X_{t} \in \overline{D} \; \text{ für alle $ t \geq 0 $ f.s., und} \\ 
& \text{$ L = (L_{t})_{t \geq 0} $ ist ein $ F_t $-adaptierter Prozess von beschränkter Variation mit} \\
& L_{t} = \int_{0}^{t} \mathsf{1}_{\{X_{s} \in \partial D \} }(s) n(X_{s}) d|L|_{s} \quad \text{ für alle $ t \geq 0 $ f.s..}
\end{aligned}  \right .
\end{equation}
Im Falle des Halbraums wird die Existenz einer eindeutigen Lösung $(X,L)$ von (\ref{eq:SDE*}) gezeigt, falls die Koeffizienten $ \sigma, b $ Lipschitz und beschränkt sind. Dabei wird eine übliche Fixpunktmethode angewendet, in dem pfadweise mithilfe von (\ref{eq:S*}) eine Iteration definiert wird. Dieser konvergiert gegen die gesuchte Lösung von (\ref{eq:SDE*}).

Für allgemeine glatt berandete Gebiete $D$ erhält man durch die  Wahl von lokalen Karten eine Transformation auf den Halbraum, so dass die Koeffizienten des transformierten Problems wiederum beschränkt und Lipschitz-stetig angenommen werden können.

\subsection{Der Halbraum (deterministisch)}

Sei $ D = \{ x \in \mathbb{R}^d : x_1 > 0 \} \subset \mathbb{R}^d $ der Halbraum von $ \mathbb{R}^d $, für $ d \geq 1 $. Sei $n$ das eindeutige (innere) Normalenvektorfeld auf $ \partial D $, so dass für alle $ x \in \partial D $ gilt: $ |n(x)| = 1 $, also $ n = e_1 \in \mathbb{R}^d $. Damit ergibt sich für das Skorokhod-Problem:

\begin{defin}
Sei $ w \in C([0, \infty); \mathbb{R}^d) $, so dass $ w(0) \in \overline{D} $. Ein Paar von Funktionen $(x,l)$ heißt \emph{Lösung des Skorokhod-Problems} $ (w, D, n) $, falls
\begin{equation}
\label{eq:S}
\left \{ \begin{aligned}
& x \in C([0, \infty); \overline{D}), \; l \in C([0, \infty); \mathbb{R}^d), \; |l|_T < \infty \; \text{für alle } T < \infty, \; \text{und} \\  
& l(t) = \int_0^t \mathsf{1}_{\{ x(s) \in \partial D \} }(s) n(x(s)) d|l|(s), \quad \text{so dass} \\
& x(t) = w(t) + l(t) \qquad \text{für alle $ t \geq 0 $}.
\end{aligned} \right .
\end{equation}
\end{defin}

Hier und im Folgenden bezeichne $x$ als die reflektierte Funktion, $w$ als die unbeschränkte oder nicht-reflektierte Funktion, und $l$ als die Lokalzeit von $x$ auf dem Rand $\partial D$. \newline

Betrachte nun den Fall $d=1$, mit $ D = (0, \infty)$ und $ \partial D = \{ 0 \}$.

\begin{lemma}
\label{sec:skorokhodlemma}
Sei $ w: [0, \infty ) \to \mathbb{R} $ stetig, $w(0) \geq 0$. Dann existiert ein eindeutiges Paar $(x,l)$ von Funktionen auf $[0, \infty)$, so dass
\begin{enumerate}
\item[(i)] $ x(t) = w(t) + l(t) $ für alle $ t \geq 0 $,
\item[(ii)] $x$ ist nicht negativ, also $ x(t) \in \overline{D} $ für alle $ t \geq 0$,
\item[(iii)] $l$ ist (schwach) monoton wachsend, stetig und $l(0) = 0$, und das zugehörige Maß $dl(s)$ verschwindet außerhalb von $ \{s \geq 0 : x(s) = 0 \} $.
\end{enumerate}
Diese Funktion $l$ ist gegeben durch
\begin{equation}
\label{sec:funktionlemma}
l(t) := \sup_{s \leq t} ( - w(s) \vee 0 ).
\end{equation}
\end{lemma}

\begin{bem}
\begin{enumerate}
\item[(1)] Hierbei bezeichnet $ (a \vee b) := \max (a,b) $.
\item[(2)] Die Bedingung in (iii), dass das Maß $dl$ außerhalb von $ \{s \geq 0 : x(s) = 0\} $ verschwindet impliziert, dass $ supp (dl) \subseteq \{s \geq 0 : x(s) = 0\} $ und ist äquivalent zu 
\begin{equation*}
l(t) = \int_0^t \mathsf{1}_{\{ x(s) = 0 \} }(s) 1 dl(s) \quad \text{bzw.} \quad \int_0^{\infty} \mathsf{1}_{\{ x(s) > 0 \} }(s) dl(s) = 0.
\end{equation*} 
\end{enumerate}
\end{bem}

\begin{proof}
\emph{Existenz:} Aufgrund der Definiton von $l$ in (\ref{sec:funktionlemma}) ist klar, dass (i) und (ii) erfüllt sind, ebenso, dass $l$ monoton wachsend, stetig und in Null verschwindet. Es verbleibt zu zeigen, dass
\begin{equation*}
\int_0^{\infty} \mathsf{1}_{\{ x(s) > \epsilon \} }(s) dl(s) = 0 \quad \text{für alle } \epsilon > 0.
\end{equation*}
Sei dazu $(t_1, t_2) \subseteq \{s \geq 0 : x(s) > \epsilon \} $ offen in $ \mathbb{R}_+ $, und bemerke $-w(s) = l(s) - x(s) \leq l(t_2) - \epsilon$ für $ t_1 \leq s \leq t_2$.
Damit erhält man 
\begin{equation*}
l(t_2) = \sup_{t_1 \leq s \leq t_2} \left(  -w(s) \vee l(t_1) \right) \leq \left( ( l(t_2) - \epsilon ) \vee l(t_1) \right) .
\end{equation*}
Also $ l(t_2) = l(t_1) $. Dies bedeutet, dass $l$ auf der Menge $\{s \geq 0 : x(s) > \epsilon \}$ konstant ist für alle $\epsilon > 0$, und somit das Maß $dl$ auf eben dieser Menge verschwindet. Insgesamt sind alle geforderten Eigenschaften gezeigt.

\emph{Eindeutigkeit:} Sei $(\tilde{x}, \tilde{l})$ ein zweites Funktionenpaar, dass die Eigenschaften (i) bis (iii) erfüllt. Dann ist $ x -\tilde{x} = \tilde{l} - l $ eine Funktion von beschränkter Variation, und mittels partieller Integration folgt:
\begin{align*}
0 \leq (x - \tilde{x} )^2 (t) & = 2 \int_0^t [x(s) - \tilde{x}(s)] d[\tilde{l}(s) - l(s)] \\
& = -2 \left[ \int_0^t x(s) d \tilde{l}(s) + \int_0^t \tilde{x}(s) dl(s) \right]  \leq 0,
\end{align*} 
da $d\tilde{l}$ und $dl$ positive Maße nach (iii) und $x, \tilde{x} $ nicht negative Funktionen nach (ii) sind. Insgesamt folgt die Behauptung.
\end{proof}

\begin{bem}
Im Fall von $d\geq 2$ betrachte $ \mathbb{R}^d_+ $ als $ \mathbb{R}_+ \times \mathbb{R}^{d-1} $. Das bedeutet, für die erste Komponente $w_1 : [0, \infty) \to \mathbb{R}$ mit $ w_1(0) \geq 0$ wird mithilfe des Lemmas~\ref{sec:skorokhodlemma} die reflektierte Funktion $x_1$ von $w_1$ mit $l_1$ als Lokalzeit auf dem Rand $\partial \mathbb{R}_+$ gefunden. Setze dann 
\begin{align*}
x(t) & = ( x_1(t), \; w_2(t), \; \dots \; , \; w_d(t)), \text{ und } \\
l(t) & = ( l_1(t), \; 0, \; \dots \; , \; 0 ) \qquad \text{ für alle $t \geq 0$}.
\end{align*}
Dann erhält man
\begin{enumerate}
\item[(i)] $x(t) = w(t) + l(t) $ für alle $t\geq 0$,
\item[(ii)] $x(t) \in \mathbb{R}^d_+$  für alle $t\geq 0$ und
\item[(iii)] $l: [0, \infty) \to \mathbb{R}^d$ ist eine stetige Funktion mit beschränkter Variation, so dass
\begin{align*}
l(t) & = \int_0^t \mathsf{1}_{\{ x_1(s) = 0 \} }(s) n(x(s)) d|l|(s) \\
& = e_1 \int_0^t \mathsf{1}_{\{ x_1(s) = 0 \} }(s) dl_1(s) + \sum_{i=2}^d e_i 0.
\end{align*}
\end{enumerate}
Also ist mit $(x,l)$ eine eindeutige Lösung des Skorokhod-Problems (\ref{eq:S}) für den d-dimensionalen Halbraum $\mathbb{R}_+^d$ gegeben. 
\end{bem}



\subsection{Stochastische Differentialgleichungen auf dem Halbraum}
\label{sec:SDE_Halbraum}

\subsubsection{Definitionen und Voraussetzungen}


Sei $(\Omega, F, P) $ ein vollständiger Wahrscheinlichkeitsraum mit einer aufsteigenden Folge von Teil-$\sigma$-Algebren $(F_t)_{t \geq 0} $ von $F$, so dass $F_t$ alle $P$-Nullmengen enthält und $ F_t = \cap_{s > t} F_s $. Weiterhin sei $ B = (B_t)_{t \geq 0} $ eine d-dimensionale $ F_{t} $-Brownsche Bewegung mit $ B_{0} = 0 $. Ferner sei $ D \subset \mathbb{R}^{d} $ der Halbraum, das heißt, $ \overline{D} = \{ x \in \mathbb{R}^{d} : x_{1} \geq 0 \} $.\newline

Sei $ x_{0} \in \overline{D} $, und seien $ \sigma^{ij} $ und $ b^{i} $ für $ i,j \in \{ 1, \dots , d \} $ Borel-messbare, reellwertige Funktionen auf $ \mathbb{R}_{+} \times \mathbb{R}^{d} $, die eine \emph{globale Lipschitzbedingung} erfüllen
\begin{align*}
& \exists  K_{\sigma^{ij}},K_{b^i} > 0, \; \forall i,j \in \{ 1, \dots , d \},  \; \forall t \geq 0 \; \text{ und } x,y \in \mathbb{R}^{d}: \\ \tag{L}
 &| \sigma^{ij}(t,x) - \sigma^{ij}(t,y) | \leq K_{\sigma^{ij}} | x - y |,  \quad | b^{i}(t,x) - b^{i}(t,y) | \leq K_{b^i} | x - y |,  
\end{align*}
und eine \emph{Wachstumsschranke} besitzen
\begin{align*}
&\exists K > 0, \; \forall i,j \in \{ 1, \dots , d \},  \; \forall t \geq 0 \\
&| \sigma^{ij}(t,x) |^{2} \leq K (1 + | x |^{2} ),  \quad | b^{i}(t,x) |^{2} \leq K (1 + | x |^{2} ). \tag{W}
\end{align*}

Das Ziel dieses Abschnitts ist die Existenz einer (pfadweise) eindeutigen Lösung des folgenden Problems:

\begin{defin}
Sei $ D \subseteq \mathbb{R}^{d} $ eine offene und zusammenhängende Menge mit glattem Rand $ \partial D $, so dass auf $ \partial D $ ein eindeutiges inneres, normiertes Normalenvektorfeld definiert ist. Sei weiterhin $ x_{0} \in \overline{D} $ und die Koeffizienten $ \sigma, b $ sind global Lipschitz-stetig und haben eine Wachstumsschranke. Dann finde ein stetiges $ F_{t} $-adaptiertes Semimartingal $ X = (X_{t})_{t \geq 0} $, das folgende Eigenschaften erfüllt:
\begin{equation}
\label{eq:SDE}
\left \{ \begin{aligned} 
& X_{t} = x_{0} + \int_{0}^{t} \sigma (s, X_{s}) dB_{s} + \int_0^t b(s,X_{s})ds + L_{t}, \\ 
& X_{t} \in \overline{D} \; \text{ für alle $ t \geq 0 $ f.s., und} \\ 
& \text{$ L = (L_{t})_{t \geq 0} $ ist ein $ F_t $-adaptierter Prozess von beschränkter Variation mit} \\
& L_{t} = \int_{0}^{t} \mathsf{1}_{\{X_{s} \in \partial D \} }(s) n(X_{s}) d|L|_{s} \quad \text{ für alle $ t \geq 0 $ f.s..}
\end{aligned} \right .
\end{equation}
\end{defin}

Der so charakterisierte Prozess $ X $ heißt der \emph{reflektierte Prozess}, oder Lösung der stochastischen Differentialgleichung mit Reflexion, der Prozess $ L $ heißt die \emph{Lokalzeit} von $ X $ auf dem Rand $ \partial D $. 

\begin{bem}
Äquivalent zu der Formulierung~(\ref{eq:SDE}) ist unter den selben Annahmen an das Gebiet $ D $ und die Koeffizienten $ b, \sigma $:

Für fast alle $ \omega \in \Omega $: $ X_{\cdot}(\omega) $ ist die erste Komponente der Lösung $ (X_{\cdot}(\omega), L_{\cdot}(\omega)) $ des Skorokhod-Problems $ (x_{0} +  \int_{0}^{\cdot} \sigma (s, X_{s}(\omega)) dB_{s}(\omega) + \int_0^{\cdot} b(s,X_{s}(\omega))ds, D, n) $.
\end{bem}

Um die Existenz und Eindeutigkeit einer Lösung von (\ref{eq:SDE}) nachzuweisen, ist das folgende Lemma hilfreich (siehe Theorem 3.2. in \cite{Andres-Diplom}).

\begin{lemma}
\label{sec:sebastianslemma}
Für festes $ T>0 $ seien $ (X_t)_t $ und $ (Y_t)_t $ für $ 0 \leq t \leq T $ zwei Lösungen von (\ref{eq:SDE}) für $ x_0, y_0 \in \mathbb{R}^d_+ $. Dann gibt es für festes $ p \geq 4 $ eine positive Konstante $ C $, die nur von $ p $ und $ T $ abhängt, so dass
\begin{equation}
\label{eq:sebastianslemma}
E \left[ \sup_{0\leq t \leq T} | X_t - Y_t |^{p} \right] \leq C | x_0 -y_0 |^{p} + C \int_0^T E \left[ \sup_{0\leq s \leq r} | X_s - Y_s |^{p} \right] dr.
\end{equation}
\end{lemma}

\begin{proof}
Sei $ T>0 $. Da $  X^i_t dL^i_t = 0 $ für alle $ t \geq 0 $ und $ 1 \leq i \leq d $, und analog $  Y^i_t d\tilde{L}^i_t = 0 $  erhält man
\begin{equation}
\label{sec:hilfe1}
\left( X^i_t - Y^i_t \right) \left( dL^i_t - d\tilde{L}^i_t \right) = - X^i_t d\tilde{L}^i_t - Y^i_t  dL^i_t \leq 0.
\end{equation}
Mithilfe der Cauchy-Schwarz-Ungleichung für die Euklidische Norm $ \| \cdot \| $ auf $ \mathbb{R}^d $ erhält man
\begin{align}
\begin{split}
\label{sec:hilfe2}
& \sum_{i = 1}^{d} \left( X^i_t - Y^i_t \right) \left( b^i(t,X_t) - b^i(t, Y_t) \right) \\
& \qquad \leq \|  X^i_t - Y^i_t \| \left( \sum_{i = 1}^{d} \left( b^i(t,X_t) - b^i(t, Y_t) \right)^2 \right)^{1/2} \\
& \qquad \leq \|  X^i_t - Y^i_t \| \left( \sum_{i = 1}^{d} K_{b^i}^2 \| X_t - Y_t \|^2 \right)^{1/2} \\
& \qquad = K_b \|  X^i_t - Y^i_t \|^2, 
\end{split}
\end{align}
wobei $ K_b := \left( \sum_{i=1}^d K_{b^i}^2 \right)^{1/2} $.

Da die Abbildung $ t \mapsto L^i_t $ für alle $ 1 \leq i \leq d $ von beschränkter Variation ist, gilt
\begin{equation}
\label{sec:hilfe3}
d \left\langle X^i - Y^i \right\rangle_t = \sum_{j = 1}^d \left( \sigma^{ij}(t,X_t) - \sigma^{ij}(t,Y_t) \right)^2 dt,
\end{equation}
wobei $ \left\langle X \right\rangle $ die quadratische Variation von dem Prozess $ X $ darstellt.

Mithilfe von It\^{o}s Formel der partiellen Integration und (\ref{sec:hilfe1}), (\ref{sec:hilfe2}), (\ref{sec:hilfe3}) und der Lipschitz-Stetigkeit der Funktionen $ \sigma^{ij} $ für $ 1 \leq i,j \leq d  $ erhält man
\begin{align*}
\begin{split}
 d \| X_t  - Y_t \|^2 & = \sum_{i = 1}^d d ( X^i_t - Y^i_t )^2 \\
 &  = \sum_{i = 1}^d \left[ 2 ( X^i_t - Y^i_t ) d ( X^i_t - Y^i_t ) +  \qvar{X^i - Y^i}_t \right] \\
&  = 2 \sum_{i = 1}^d ( X^i_t - Y^i_t ) \left( b^i(t,X_t) - b^i(t, Y_t) \right) dt \\
& \quad + 2 \sum_{i = 1}^d ( X^i_t - Y^i_t ) (dL^i_t - d \tilde{L}^i_t) \\
& \quad + 2 \sum_{i,j = 1}^d ( X^i_t - Y^i_t ) \left( \sigma^{ij}(t,X_t) - \sigma^{ij}(t,Y_t) \right) dB^j_t \\
& \quad + \sum_{i,j = 1}^d \left( \sigma^{ij}(t,X_t) - \sigma^{ij}(t,Y_t) \right)^2 dt \\
& \leq ( 2  K_b + K_{\sigma}^2 ) \| X_t - Y_t \|^2 dt \\
& \qquad \qquad + 2 \sum_{i,j = 1}^d ( X^i_t - Y^i_t ) \left( \sigma^{ij}(t,X_t) - \sigma^{ij}(t,Y_t) \right) dB^j_t,
\end{split}
\end{align*}
wobei $ K_{\sigma} := \left( \sum_{i,j=1}^d K_{\sigma^{ij}}^2 \right)^{1/2} $. Mit anderen Worten:
\begin{align}
\label{eq:lemma-eindeutigkeit}
\begin{split}
\| X_t - Y_t \|^2 & \leq \| x_0 - y_0 \|^2 + (2 K_b + K_{\sigma}^2) \int_0^t \| X_s - Y_s \|^2 ds \\
& \quad + 2 \sum_{i,j = 1}^d \int_0^t ( X^i_s - Y^i_s ) \left( \sigma^{ij}(s,X_s) - \sigma^{ij}(s,Y_s) \right) dB^j_s.
\end{split}
\end{align}


Im Folgenden bezeichne $ C_i, \; i = 1, \dots , 6, $ positive Konstanten, die nur von $ T $ und $ p $ abhängen.

Sei $ p \geq 2 $ fest. Durch zweimaliges Anwenden der Hölder-Ungleichung auf die rechte Seite von (\ref{eq:lemma-eindeutigkeit}) erhält man für $ t \in [0,T] $:
\begin{align}
\label{eq:lemma-fastfertig}
\begin{split}
\| X_t - Y_t \|^{2p} & \leq C_1 \| x_0 - y_0 \|^{2p} + C_2 \left( \int_0^t \| X_s - Y_s \|^2 ds  \right)^p \\
&  \quad + C_3 \left( \sum_{i,j =1}^d \int_0^t ( X^i_s - Y^i_s ) \left( \sigma^{ij}(s,X_s) - \sigma^{ij}(s,Y_s) \right) dB^j_s  \right)^p \\
&  \leq C_1 \| x_0 - y_0 \|^{2p} + C_4 \int_0^t \| X_s - Y_s \|^{2p} ds \\
& \quad +  C_3 \left( \sum_{i,j =1}^d \int_0^t ( X^i_s - Y^i_s ) \left( \sigma^{ij}(s,X_s) - \sigma^{ij}(s,Y_s) \right) dB^j_s  \right)^p.
\end{split}
\end{align}

Der letzte Term auf der rechten Seite von (\ref{eq:lemma-eindeutigkeit}) ist ein stetiges lokales Martingal, das in der Null verschwindet. Somit gilt für die quadratische Variation:
\begin{align}
\label{eq:lemma_qvar}
\begin{split}
& \qvar{ 2 \sum_{i,j = 1}^d \int_0^{\cdot} ( X^i_s - Y^i_s ) \left( \sigma^{ij}(s,X_s) - \sigma^{ij}(s,Y_s) \right) dB^j_s}_t \\
& \quad = 4\sum_{i,j,k = 1}^d \int_0^t ( X^i_s - Y^i_s ) \left( \sigma^{ij}(s,X_s) - \sigma^{ij}(s,Y_s) \right) \\
& \hspace*{3cm} \times ( X^i_s - Y^i_s ) \left( \sigma^{ik}(s,X_s) - \sigma^{ik}(s,Y_s) \right) d \qvar{B^j,B^k}_s \\
& \quad =  4\sum_{i,j = 1}^d \int_0^t ( X^i_s - Y^i_s )^2 \left( \sigma^{ij}(s,X_s) - \sigma^{ij}(s,Y_s) \right)^2 ds \\
& \quad \leq 4 K_{\sigma}^2 \int_0^t \| X_s - Y_s \|^4 ds,
\end{split}
\end{align}
wobei erneut die Lipschitz-Stetigkeit der Koeffizienten $ \sigma^{ij} $ und die Eigenschaft der Brownschen Bewegung $ \qvar{B^i, B^j}_ t = \delta_{i,j} t $ ausgenutzt worden ist. Mithilfe der Burkholder-Ungleichung angewendet auf (\ref{eq:lemma-fastfertig}) und (\ref{eq:lemma_qvar}) folgt:

\begin{align*}
& E \left[ \sup_{0 \leq t \leq T}  \| X_t - Y_t \|^{2p} \right] \\
& \quad  \leq C_1 \| x_0 -y_0 \|^{2p} + C_4 E \left[ \sup_{0 \leq t \leq T} \int_0^t \| X_s -Y_s \|^{2p} ds  \right] \\
& \qquad + C_3 E \left[ \sup_{0 \leq t \leq T} \left(   \sum_{i,j = 1}^d \int_0^{t} ( X^i_s - Y^i_s ) \left( \sigma^{ij}(s,X_s) - \sigma^{ij}(s,Y_s) \right) dB^j_s \right)^{p}  \right] \\
& \quad \leq C_1 \| x_0 -y_0 \|^{2p} + C_4 E \left[  \int_0^T \| X_s -Y_s \|^{2p} ds  \right] + C_5 E \left[ \left( \int_0^T \| X_s - Y_s \|^4 ds \right)^{p/2} \right].
\end{align*}

Mithilfe der Hölder-Ungleichung mit Hölderexponenten $ p/2 $ und $ p/(p-2) $ für $ p>2 $ und $ 1 $ und $ \infty $ für $ p=2 $ und dem Satz von Fubini folgt aufgrund der Nicht-Negativität des Integranden:
\begin{align*}
\label{eq:lemma-letzte_gleichung}
\begin{split}
 E \left[ \sup_{0 \leq t \leq T}  \| X_t - Y_t \|^{2p} \right] & \leq C_1 \| x_0 -y_0 \|^{2p} + C_6 E \left[  \int_0^T \| X_s -Y_s \|^{2p} ds  \right] \\
 & \leq  C_1 \| x_0 -y_0 \|^{2p} + C_6 \int_0^T E \left[ \sup_{0 \leq t \leq r}  \| X_t - Y_t \|^{2p}  \right] dr.
\end{split}
\end{align*}
Das beendet den Beweis.

\end{proof}

\begin{bem}
Die Voraussetzungen des vorangegangenen Lemmas können dahingehend abgeändert werden, dass $ X $ von der Form Semimartingal $ + $ Lokalzeit ist und die Koeffizienten $ \sigma, b $ Lipschitz-stetig sind. Dies wird später genutzt, um die Fixpunktmethode anzuwenden.
\end{bem}

Im Beweis der pfadweisen Eindeutigkeit wird eine Teilaussage des Lemmas~(\ref{eq:sebastianslemma}) verwendet.

\begin{kor}
\label{sec:Korollar von sebastianslemma}
Für $ p \geq 4 $ und $ T>0 $ gilt auch mit den gleichen Voraussetzungen wie in Lemma~(\ref{sec:sebastianslemma}) eine schwächere Aussage für $ t \in [0,T] $:
\begin{align}
\begin{split}
 E \left[  \| X_t - Y_t \|^{p} \right] & \leq   C \| x_0 -y_0 \|^{p} + C \int_0^T E \left[  \| X_t - Y_t \|^{p}  \right] ds.
\end{split}
\end{align}
\label{eq:lemma-bemerkung}
\end{kor}



\subsubsection{Eindeutigkeit und Existenz der SDE mit Reflexionen}

Um im Folgenden die (pfadweise) Eindeutigkeit und die Existenz der SDE mit normaler Reflexion auf dem Halbraum $ D = \mathbb{R}^d_+ $ zu beweisen, wird eine Fixpunktmethode angewandt. Dabei betrachtet man für festes $ T > 0 $ und $ p \geq 4 $ den Raum $ Z $ der stetigen, $ F_t $-adaptiven Prozesse $ X_t $ mit $ t \in [0,T] $ und 
\begin{equation}
\| X \|_{p,T} := E \left[ \sup_{0 \leq t \leq T} | X_t |^p \right]^{1/p} < \infty ,
\end{equation}
und definiert für $ n \in \mathbb{N} $ eine Picard-Lindelöf-Iteration mittels
\begin{equation}
X^{(n)}_t \mapsto x_{0} + \int_0^t \sigma (s, X_{s}^{(n)}) dB_s + \int_0^t b(s, X_{s}^{(n)}) ds + L^{(n+1)}_t := F \left( X^{(n)}_t \right),
\end{equation}
wobei $ F : Z \to Z $ durch eine geeignete Wahl von $ T $ zur Kontraktion wird. Mithilfe des Banachschen Fixpunktsatzes folgt schließlich die Existenz und Eindeutigkeit.

Allerdings kann man die Eindeutigkeit der Lösung von (\ref{eq:SDE}) für eine größere Klasse an stochastischen Prozessen zeigen. Denn durch die Wahl von geeigneten Stoppzeiten ist die Voraussetzung an die Integrabilität der Prozesse obsolet, so dass schließlich mit dem Korollar (\ref{sec:Korollar von sebastianslemma}) und dem Lemma von Gronwall die pfadweise Eindeutigkeit folgt.
\begin{thm}
Falls die Koeffizienten $ \sigma , b $ eine globale Lipschitz-Bedingung (L) erfüllen und eine Wachstumsschranke (W) besitzen, so existiert eine pfadweise eindeutige $ F_{t} $-adaptierte Lösung von (\ref{eq:SDE}) für jedes $ x_{0} \in \overline{D} $.
\end{thm}


\begin{proof}
\emph{Eindeutigkeit:} Sei $ p \geq 4 $ fest. Seien $ X =(X_{t})_{t \geq 0} $ und $ \tilde{X} = (\tilde{X_{t}})_{t \geq 0} $ zwei $ F_t $-adaptierte Lösungen von (\ref{eq:SDE}) für $ x_0 \in \overline{D} $. Um den Erwartungswert von $ | X_t - \tilde{X}_t |^{p} $ zu nehmen, definiere folgende $ F_t $-Stoppzeiten
\begin{equation*}
T_k = \inf \left\lbrace t \geq 0 : \int_0^t \| X_s - \tilde{X}_s \|^{p} ds + |L|_t + |\tilde{L}|_t = k  \right\rbrace , \; \text{ mit } T_k \nearrow \infty \text{ für } k \to \infty.
\end{equation*}
Betrachte nun die gestoppten Prozesse $ | X_{t \wedge T_k} - \tilde{X}_{t \wedge T_k} |^{p} $ und nehme den Erwartungswert. Dann erhält man mit  Korollar (\ref{sec:Korollar von sebastianslemma}):
\begin{align}
\label{eq:Thm-Stoppzeiten}
\begin{split}
E[ \| X_{t \wedge T_k} - \tilde{X}_{t \wedge T_k} \|^{p} ] & \leq C \int_0^T E \left(  \|  X_{t \wedge T_k} - \tilde{X}_{t \wedge T_k} \|^{p}  \right] dt.
\end{split}
\end{align}
Mit dem Lemma von Gronwall folgt für alle $ t \geq 0 $ und $ k \in \mathbb{N} $:
\begin{equation*}
E[ | X_{t \wedge T_k} - \tilde{X}_{t \wedge T_k} |^{p} ] = 0.
\end{equation*}
Somit gilt dies auch ohne die Stoppzeiten $ T_k $. Aufgrund der Stetigkeit der beiden Prozesse $ X, \tilde{X} $ ergibt sich die Ununterscheidbarkeit:
\begin{equation*}
X_t = \tilde{X}_t \text{ für alle $ t \geq 0  \quad P $-f.s.. }
\end{equation*}

\emph{Existenz}: Um die Existenz zu beweisen, definiere iterativ eine Folge $ (X^{(n)})_{n \in \mathbb{N}} $ von $ \overline{D} $-wertigen Prozessen durch
\begin{align*}
X_{t}^{(0)} & := x_{0},\\
X_{t}^{(n)} & := x_{0} + \int_0^t \sigma (s, X_{s}^{(n-1)}) dB_s + \int_0^t b(s, X_{s}^{(n-1)}) ds + L^{(n)}_t, \quad n \geq 1, t \geq 0.
\end{align*}
Dabei wird der Prozess $ X^{(n)} $ mithilfe des deterministischen Skorokhod-Problems (\ref{eq:S}) pfadweise für fast alle $ \omega \in \Omega $ konstruiert. Das bedeutet, dass $ X^{(n)}_{\cdot}(\omega) $ die erste Komponente der Lösung $ ( X^{(n)}_{\cdot}(\omega), L^{(n)}_{\cdot}(\omega) ) $ des Skorokhod-Problems $ (x_{0} + \int_0^{\cdot} \sigma (s, X_{s}^{(n-1)}) dB_s + \int_0^{\cdot} b(s, X_{s}^{(n-1)}) ds, D, n ) $ ist.

Nun sind die Voraussetzungen für den Banachschen Fixpunktsatz zu beweisen. Sei $ T >0 $ und $ p \geq 4 $ fest.

(1) Der Raum $ ( Z , \| \cdot \|_{p,T} ) $ ist ein Banachraum.
Sei $ X^{(n)} $ eine Cauchy-Folge in $ Z $, das bedeutet,
\begin{equation}
\label{eq:thm-ex und eind-cf}
E \left[ \sup_{0 \leq t \leq T} \left| X^{(n)}_t - X^{(m)}_t \right|^p  \right] \to 0, \quad n,m \to \infty.
\end{equation}
Es existiert eine Teilfolge $ X^{(n_k)} $, so dass für $ P $-f.a. $ \omega \in \Omega $ gilt:
\begin{equation*}
\sup_{0 \leq t \leq T} \left| X^{(n_k)}_t - X^{(n_l)}_t \right|^p \to 0, \qquad l,k \to \infty.
\end{equation*}
Aufgrund der Vollständigkeit von $ \mathbb{R} $ existiert ein stetiger, adaptierter Prozess $ X $, so dass:
\begin{equation*}
Y_k := \sup_{0 \leq t \leq T} \left| X_t^{(n_k)} - X_t \right|^p \to 0 \qquad P\text{-f.s}, \quad k \to \infty.
\end{equation*}
Mit der Dreiecksungleichung und (\ref{eq:thm-ex und eind-cf}) ist $ Y_k $ eine Cauchy-Folge in $ L^p(\mathbb{R}) $, die $ P $-f.s. gegen 0 konvergiert. Also $ \| X^{(n)} - X \|_{p,T} \to 0 \qquad n\to \infty $.
\newline

(2) Die Folge $ X^{(n)} $ ist wohldefiniert. 
Das stochastische Integral der approximierenden Folge existiert aufgrund der Wachstumsschranke (W). Rechne dazu aus:
\begin{align*}
E \left[ \int_0^T \left| \sigma(s,X_{s}^{(n)}) \right|^2 ds \right] & \leq E \left[ \int_0^T K_W^2 \left(1 + \left|X_{s}^{(n)}\right|^2 \right) ds \right] \\
&  \leq K_W^2 E \left[ \int_0^T \left(1 + \sup_{t \leq T} \left|X_{t}^{(n)}\right|^2 \right) ds \right] \\
&  \leq K_W^2 T \left( 1 + E \left[ \sup_{t \leq T} \left|X_{t}^{(n)}\right|^2 \right] \right) \\
&  \leq K_W^2 T \left( 1 + \| X^{(n)} \|_{p,T} \right).
\end{align*}
Ferner erhält man mit der Bemerkung nach Lemma~(\ref{sec:sebastianslemma}) und indem man die Definition der Folge ausnutzt:
\begin{align*}
\begin{split}
\|  X^{(n+1)} - X^{(n)} \|_{p,T}^p & = E \left[ \sup_{0 \leq t \leq T} \left| X^{(n+1)}_t - X^{(n)}_t \right|^p  \right] \\
& \leq C \int_0^T E \left[ \sup_{0 \leq t \leq T} \left| X^{(n)}_t - X^{(n-1)}_t \right|^p  \right] = CT \|  X^{(n)} - X^{(n-1)} \|_{p,T}^p.
\end{split}
\end{align*}
Da $ Y := 0 $ eine Lösung von (\ref{eq:SDE}) zum Startwert $ y_0 := 0 $ ist, erhält man die Wohldefiniertheit der approximierenden Folge, also $ F : Z \to Z $ :
\begin{equation*}
\|  F ( X^{(n)} ) \|_{p,T}^p = \|  X^{(n+1)} \|_{p,T}^p   \leq CT \|  X^{(n)}  \|_{p,T}^p.
\end{equation*}

(3) Für $ T $ klein genug ist $ F $ eine Kontraktion auf $ Z $. Seien dafür $ X,Y \in Z $. Mit der Bemerkung nach Lemma~(\ref{sec:sebastianslemma}) folgt:
\begin{align*}
\begin{split}
\| F (X) - F (Y) \|^p_{p,T} & = E \left[ \sup_{0 \leq t \leq T} | F(X)_t - F(Y)_t |^p  \right] \\
&  \leq C \int_0^T E \left[ \sup_{0 \leq t \leq r} |X_t - Y_t |^p  \right] dr = CT \|  X - Y \|_{p,T}^p.
\end{split}
\end{align*}

Wähle nun $ T $, so dass $ CT < 1 $ ist.

(4) Nach dem Banachschen Fixpunktsatz hat die Kontraktion $ F $ auf dem Banachraum $ Z $ genau einen Fixpunkt $ (X_t)_t \in Z $ für $ 0 \leq t \leq T $. Dieser ist nach Konstruktion eine Lösung von (\ref{eq:SDE}).

(5) Konstruiere die Lösung $ X $ auf $ \mathbb{R}_+ $ nun iterativ,  da $ T>0 $ unabhängig vom Startwert gewählt werden kann. Definiere für $ t \in [kT,(k+1)T] $ mit $ k \in \mathbb{N} $, die Lösung $ (X_t)_t $ zum Startwert $ X_{kT} $.

\end{proof}

\begin{bem}
Bei dem Beweis der (starken) Eindeutigkeit einer Lösung von (\ref{eq:SDE}) wird keine Voraussetzung an die Integrabilität der Lösung gemacht. Durch die Verwendung der Stoppzeiten $ T_k $ in (\ref{eq:Thm-Stoppzeiten}) gilt die Eindeutigkeit auch für nicht integrierbare Lösungen $ X $.
\end{bem}


\subsection{Lokalisation}
\label{sec:lokalisation}

Sein nun $ D_0 $ eine offene, zusammenhängende Teilmenge des $ \mathbb{R}^d $ mit Abschluss $ D $. Der Rand $ \partial D $ sei glatt, so dass an jedem Punkt des Randes die innere Normale $ n $ definiert ist. Um auf diesem Gebiet $ D $ das Skorokhod-Problem (\ref{eq:S}) oder (\ref{eq:SDE}) zu lösen, wird im folgenden Abschnitt die Lokalisations-Technik von Anderson und Orey eingeführt. Dafür nehmen wir an, dass ein System von lokalen Koordinaten mit den notwendigen Eigenschaften existiert; also das Problem des glatten Randes lokal auf den bereits gelösten Halbraum transformiert werden kann. Die Existenz setzt gewisse Anforderungen an die Glattheit des Randes und die innere Normale voraus (siehe dazu \cite{Anderson-Orey}). 

Sei dazu $ \left\lbrace U_0, U_1, \dots \right\rbrace $ eine abzählbare oder endliche Familie relativ offener Teilmengen von $ D $, die $ D_0 $ überdecken, genannt Kartenumgebungen. Jedes $ U_m $ sei mit einem Koordinatensystem ausgestattet, das heißt, mit einer Kartenabbildung $ u_m: U_m \to \mathbb{R}^d $, die jedem Punkt $ x \in U_m $ die Koordinaten $ u_m(x) = (u_m^1(x), u_m^2(x), \dots , u_m^d(x) ) $ zuordne und die folgenden Eigenschaften erfülle:

\begin{enumerate}
\item[(i)] $ U_0 \subseteq D_0 $ und die zu $ U_0 $ assoziierten Koordinaten entsprechen den original Euklidischen Koordinaten. Für $ U_m $ mit $ m > 0 $ sind die Karten $ u_m : U_m \to u_m(U_m) \subseteq \mathbb{R}^d $ $C^2$-Diffeomorphismen und es gilt:
\begin{equation*}
U_m \cap \partial D = \left\lbrace x \in U_m : u_m^1(x) = 0 \right\rbrace, \quad U_m \cap D_0 = \left\lbrace x \in U_m : u_m^1(x) > 0 \right\rbrace.
\end{equation*}
\item[(ii)] Es existiert eine positive Konstante $ \theta $ und für jedes $ x \in D $ ein Index $ m(x) \in \mathbb{N} $, so dass $ \overline{ B_{\theta}(x) } \subseteq U_{m(x)}  $.
\item[(iii)] Für jedes $ m > 0 $ und $ i \in \left\lbrace 1, ... , d \right\rbrace $ ist $\left\langle \nabla u_m^i(x) , n(x)\right\rangle = \delta_{1i} $ für alle $ x \in U_m \cap \partial D $. Die innere Normale von $ D $ wird demnach auf $e_1$ den Standard-Einheitsvektor abgebildet.
\item[(iv)] Für jedes $ m \geq 0 $ und $ i \in \left\lbrace 1, ... , d \right\rbrace $ sind die Funktionen
\begin{align}\label{eq:lokalisation-b}
& b_m^i : \mathbb{R}_+ \times U_m \to \mathbb{R},\quad  x \mapsto \left\langle \nabla u^i(x) , b(t,x) \right\rangle + \frac{1}{2} \sum_{k,j = 1}^d \frac{\partial u^i}{\partial x^k \partial x^j}(x) a^{kj}(t,x),\\
& \sigma_m^i : \mathbb{R}_+ \times U_m \to \mathbb{R}^d, \quad  x \mapsto \sum_{k,j = 1}^d \frac{\partial u^i}{\partial x^j}(x) \sigma^{jk}(t,x) e_k, \label{eq:lokalisation-sigma}
\end{align}
gleichmäßig beschränkt und Lipschitz-stetig. Hierbei ist $ a := \sigma \sigma^* $. Mit anderen Worten: Für jedes $ t,m \geq 0$ und $ i \in \left\lbrace 1, ... , d \right\rbrace $ gilt:
\begin{equation}
\sup \left[ \sup_{x \in U_m} \left( |b^i_m(t,x) | + \| \sigma_m^i(t,x) \| \right) \right] < \infty,
\end{equation}
und es existiert eine positive Konstante $c$, unabhängig von $m,t$ und $i$, so dass gilt:
\begin{equation}
| b_m^i(t,x) - b_m^i(t,y) | + \| \sigma_m^i(t,x) - \sigma_m^i(t,y) \| \leq c \| x - y \|
\end{equation}
für alle $x$ und $y$ in $U_m$.
\item[(v)] Es existiert eine positive Konstante $ d $, so dass für jedes $ U_m $ die assoziierte Kartenabbildung $ u_m $ folgende Bedingung erfüllt:
\begin{equation*}
\frac{1}{d} \| u_m(x) - u_m(y) \| \leq \| x - y \| \leq c \| u_m(x) - u_m(y) \|
\end{equation*}
für alle $x$ und $y$ in $U_m$.
\end{enumerate}

Die Konstruktion des reflektierten Prozesses erfolgt nun schrittweise ausgehend vom Startpunkt $ x \in D $. Der Ball von Radius $ \theta $ liegt nun in einer Kartenumgebung $ U_{m(x)} $ mit assozierter Kartenabbildung $ u_{m(x)} $. Solange sich der Prozess in diesem Ball aufhält, transformiert die Karte den Ball von Radius $ \theta $ um $ x $ auf den Halbraum, auf dem für den transformierten Prozess ein refelektierter Prozess generiert werden kann. Verlässt der so konstruierte Prozess $X$ den Ball mit Radius $ \theta $ um den Startpunkt $x$, so existiert erneut ein Ball mit Radius $ \theta $ um den Trefferpunkt des Prozesses mit dem Rand des Balles. Nun transformiert man das Problem wieder auf den Halbraum, möglicherweise mit einer anderen Karte. So verfährt man iterativ.

Genauer bedeutet dies: Definiere für den reflektierten Prozess $X^x$ in $D$ mit Startwert $ x \in D $ eine Sequenz von Stoppzeiten $ ( \tau_l )_{l \ge 0} $ durch:
\begin{equation*}
\tau_0 := 0, \qquad \tau_{l+1} := \left\lbrace t > \tau_l : dist ( X_t^x , X_{\tau_l}^x ) > \theta \right\rbrace, \quad l \ge 0.
\end{equation*} 
Dabei hängt $ \tau_l $ von dem Startwert $x$ ab, und es gilt $X_t^x \in U_{m_l} $ für $ t \in [\tau_l , \tau_{l+1} ] $ mit $ m_l := m( X_{\tau_l}^x)$.

Mithilfe von It\^{o}s-Formel der partiellen Integration und (\ref{eq:lokalisation-b}), (\ref{eq:lokalisation-sigma}) erhält man für $ i \in \left\lbrace 1, ... , d \right\rbrace $ und $ t \in [\tau_l , \tau_{l+1} ] $:
\begin{equation*}
\begin{split}
 u_{m_l}^i(X_t^x) & = u_{m_l}^i(X_{\tau_l}^x) \\
& \quad + \int_{\tau_l}^t \left( \left\langle \nabla u_{m_l}^i(X_s^x) , b(s,X_s^x) \right\rangle + \frac{1}{2} \sum_{k,j = 1}^d \frac{\partial u_{m_l}^i}{\partial x_k \partial x_j}(X_s^x) a_{kj}(s,X_s^x) \right) ds \\
& \quad + \sum_{k,j = 1}^d \int_{\tau_l}^t  \frac{\partial u_{m_l}^i}{\partial x_j}(X_s^x) \sigma_{jk}(s,X_s^x) dB_s^k + e_1 \left( L_t^x - L_{\tau_l}^x \right) \\
& = u_{m_l}^i(X_{\tau_l}^x) + \int_{\tau_l}^t b_{m_l}^i (s,X_{s}^x) ds + \int_{\tau_l}^t \sigma_{m_l}^i (s,X_s^x) dB_s + e_1 \left( L_t^x - L_{\tau_l}^x \right),
\end{split}
\end{equation*}
wobei $ L_t^x $ die Lokalzeit des Prozesses $X^x$ auf dem Rand $ \partial D $ darstellt. 

\begin{bem}
Der Prozess $ u_{m_l}(X_t^x) $ ist die  Lösung des transformierten Skorokhod-Problems (\ref{eq:SDE}) auf dem Halbraum und kann wie im Abschnitt~\ref{sec:SDE_Halbraum} gelöst werden. Die entsprechende Rücktransformation $ X_t^{x} $ ist dann die Lösung des Skorokhod-Problems (\ref{eq:SDE}) auf $D$. 
\end{bem}

Durch die Wahl der lokalen Koordinaten könnte die für den Halbraum gezeigte (pfadweise) Eindeutigkeit verloren gehen. Dies ist aber nicht der Fall, wie  P. L. Lions und A. S. Sznitman in \cite{Lions-Sznitman} zeigen.

\subsection{Ausblick}

Der hier vorgestellte Lösungsansatz für stochastische Differentialgleichungen mit Reflexionen auf ausreichend regulären Gebieten ist nicht der theoretisch optimale, dennoch ist er durch die Konstruktion auf dem Halbraum vergleichsweise anschaulich. Es existieren zahlreiche Arbeiten zu diesem Thema, die weitaus weniger Regularität des Gebietes fordern, zu nennen wäre hier H. Tanaka \cite{Tanaka}, der das Skorokhod-Problem (\ref{eq:SDE}) auf konvexen Gebieten löst, oder P. L. Lions und A. S. Sznitman \cite{Lions-Sznitman}, die das Skorokhod-Problem (\ref{eq:SDE}) sowohl für normale als auch schräge Reflexion für glatt berandete Gebiete und auch etwas allgemeinere Gebiete, die eine geeignete Definition der Normalen zulassen, lösen.

Die Ansätze haben gemeinsam, dass der Rand so regulär sein muss, dass ein Normalenvektorfeld auf dem Rand definiert werden kann, das gewisse Eigenschaften erfüllt. Für allgemeinere Gebiete wie zum Beispiel Lipschitz-Gebiete ist eine andere Lösungsmethode hilfreicher. Im folgenden Kapitel wird die reflektierte Brownsche Bewegung mittels der allgemeinen Theorie der Dirichlet Formen hergeleitet.


\newpage


\section{Reflektierte Brownsche Bewegung auf Lipschitz Gebieten}
\label{sec:Reflektierte-Brownsche-bewegung-auf-Lipschitz-Gebieten}


Ziel dieses Kapitels ist der Nachweis der Existenz der reflektierten Brownschen Bewegung (engl.: reflected Brownian motion, RBM) und der zugehörigen Lokalzeit auf dem Rand eines beschränkten Lipschitz Gebietes. Damit wird im letzten Abschnitt das Neumann-Randwertproblem auf beschränkten Lipschitz Gebieten in $ \mathbb{R}^d $ für $ d \geq 3 $ gelöst. Der Existenzbeweis für die RBM und ihre Lokalzeit wurden von R. F. Bass und P. Hsu \cite{Lions-Sznitman} geführt, und basiert auf der allgemeinen Theorie der Dirichlet Formen, wie sie von M. Fukushima, Y. Oshima und M. Takeda in \cite{Fukushima} eingeführt worden ist.\\

Um im Folgenden einen kurzen Überblick über die Resultate und Methoden dieser Arbeit zu geben,  sei $ D \subset \mathbb{R}^d $ ein beschränktes Lipschitz Gebiet und bezeichne mit  $ \mathcal{E} $, die zu der RBM assoziierte Dirichlet Form auf $ L^2(D) $. Zunächst wird in Theorem~\ref{sec: Thm-MP} die Existenz der RBM auf $ \overline{D} $ gezeigt, dabei bezeichne $ \overline{D} $ den Euklidischen Abschluss von $ D $ in $ \mathbb{R}^d $. Genauer wird die Existenz eines $ \overline{D} $-wertigen, stetigen, normalen, starken Markov-Prozess gezeigt, der zu der Dirichlet Form $ \mathcal{E} $ assoziiert ist. Um die allgemeine Theorie der Dirichlet Formen anwenden zu können, muss gezeigt werden, dass die assoziierte Dirichlet Form
\begin{equation*}
\mathcal{E}(u,v) = \frac{1}{2} \int_{D} \nabla u(x) \nabla v (x) dx
\end{equation*} 
regulär ist und die lokale Eigenschaft auf beschränkten Lipschitz Gebieten erfüllt. Damit erhält man eine quasi-überall definierte Version eines stetigen, normalen, starken Markov-Prozesses auf $ \overline{D} $. Mithilfe der Übergangskerne $ p(t,x,y) $ der RBM kann man diesen auf ganz $ \overline{D} $ erweitern.

In Theorem~\ref{sec:Thm-Lokalzeit} wird die Existenz eines eindeutigen positiven, stetigen, additiven Funktionals gezeigt, was auch als Lokalzeit des Randes bezeichnet wird. Dazu muss bewiesen werden, dass das $ (d-1) $-dimensionale Oberflächenmaß $ \sigma $ des Randes $ \partial D $ endliches Energie-Integral bezüglich $ \mathcal{E} $ besitzt. Mit der Theorie der Dirichlet Formen folgt hier die entsprechende Aussage wieder für quasi-alle Punkte in $ \overline{D} $, die aber aufgrund der ausreichenden Regularität der betrachteten Objekte erweitert werden kann.

In Theorem~\ref{sec:Thm-NRWP} wird mithilfe von Theorem~\ref{sec: Thm-MP} und Theorem~\ref{sec:Thm-Lokalzeit} eine schwache Lösung des Neumann-Randwertproblems (NRWP) durch
\begin{equation*}
u(x) = \lim_{t \to \infty} \frac{1}{2} E^x \left[ \int_0^t f(X_s) dL_s \right]
\end{equation*}
angegeben, wobei $ f \in B(\partial D) $ die Randwerte und $ L $ die Lokalzeit der RBM auf dem Rand $ \partial D $ darstellt. Da D. S. Jerison und C. E. Kenig in \cite{Jerison-Kenig} bereits für beschränkte Lipschitz Gebiete die Existenz und Eindeutigkeit einer schwachen Lösung des Neumann-Randwertproblems gezeigt haben, bleibt nachzurechnen, dass $ u $ die Eigenschaften einer schwachen Lösung erfüllt.

\subsection{Definitionen und Voraussetzungen}

Im Falle von glatt berandeten Gebieten ist die Existenz der reflektierten Brownschen Bewegung bekannt. Sie kann explizit konstruiert werden:

Sei $ D $ ein beschränktes Gebiet in $ \mathbb{R}^d \; (d \geq 3) $ mit $ C^3 $-Rand. Der Laplace-Operator und die innere Normalenableitung auf dem Rand $ \partial D $ werden mit $ \Delta $ und $ \partial / \partial n $ bezeichnet.

Betrachte die folgende \emph{Wärmeleitungsgleichung mit Neumann Randbedingungen}:

\begin{equation}
\label{eq:PDE}
\left \{ \begin{aligned}
& \frac{\partial}{\partial t} p(t,x,y)  = \frac{1}{2} \Delta_x p(t,x,y),  & x \in D, \;  y \in \overline{D}; \\
& \frac{\partial}{\partial n_x} p(t,x,y)  = 0,  & x \in \partial D, \;  y \in \overline{D}; \\
& \lim_{t \to 0} p(t,x,y)  = \delta_x (y),  & x \in \overline{D}, \;  y \in \overline{D}.
\end{aligned}\right .
\end{equation}
(Hierbei bezeichnet der Index $ x $, dass die Operation auf die $ x $ Variable angewendet wird.) Im Falle von $ C^3 $-Rand kann diese Wärmeleitungsgleichung gelöst werden, und auf dem Standardpfadraum über $ \overline{D} $ kann ein stetiger, starker Markov-Prozess $ \{ C([0, \infty); \overline{D}), F, F_t, P^x, X_t , t \geq 0 \} $  konstruiert werden, dessen Übergangskerne $ p(t,x,y) $ sind. Dieser Prozess wird die \emph{Standard-RBM} auf $ D $ genannt. 

Eine äquivalente Definition der RBM auf beschränkten Gebieten mit $ C^3 $-Rand ist die folgende stochastische Differentialgleichung mit reflektierender Randbedingung:
\begin{equation*}
dX_t = dB_t + n(X_t) dL_t,
\end{equation*}
wobei $ B $ eine Standard-Brownsche Bewegung auf $ \mathbb{R}^d $ ist, $ n(a) $ die innere Normale an $ a \in \partial D $ bezeichnet, und $ L $ ein stetiges, additives Funktional von $ X $ ist, das auch die Lokalzeit der RBM auf dem Rand genannt wird. (Für eine genauere Darstellung der Zusammenhänge siehe \cite{Hsu}.)

\begin{bem}
Beachte hierbei, dass im Bezug auf den ersten Teil dieser Arbeit eine leicht veränderte Definition der Lokalzeit verwendet wird. Die Lokalzeit hier entspricht der totalen Variation der Lokalzeit im ersten Teil.\\
\end{bem}


Nun, wie im Folgenden sei $ D $ ein beschränktes Lipschitz Gebiet in $ \mathbb{R}^d $ für $ d \geq 3 $. Analog zu (\ref{eq:PDE}) kann man den Übergangskern der RBM $ p(t,x,y) $ als den Wärmeleitungskern für $  \Delta /2 $ auf $ D $ mit Neumann Randbedingungen charakterisieren.

Mittels der üblichen $ L^2 $-Methode aus der Theorie der elliptischen partiellen Differentialgleichungen kann die Existenz von $ p(t,x,y) $ auf beliebigen beschränkten Gebieten ohne jede Regularitätsbedingung an das Gebiet gezeigt werden (siehe  \cite{Gilbarg-Trudinger} für genauere Ergebnisse). Dabei erfüllt die Familie $ \{ p(t,x,y); t >0, \; x,y \in D  \} $ die Eigenschaften eines glatten Übergangskerns auf $ D $.

\begin{defin}
Eine Familie von Funktionen $ \{ p(t,x,y); t >0, \; x,y \in D  \} $ heißt ein \emph{glatter Übergangskern} auf $ D $, falls die folgenden Eigenschaften erfüllt sind:

\begin{enumerate}
\item[(1)] $ p(t,x,y) \geq 0 \quad \text{ für } t>0, \; x,y \in D; $
\item[(2)] $ \int_D p(t,x,y) dy = 1 \quad \text{ für } t>0, \; x \in D; $
\item[(3)] $ p(t+s,x,y) = \int_D p(t,x,z)p(s,z,y) dz \quad \text{ für } t,s>0, \; x,y \in D; $
\item[(4)] $ p(t,x,y) \text{ ist glatt auf } (0, \infty) \times D \times D $ und symmetrisch in $ (x,y) \in D \times D $.
\end{enumerate}
\end{defin}

Die Eigenschaft (2) heißt die \emph{Konservativität} von $ p(t,x,y) $ und die Eigenschaft (3) ist auch als die \emph{Chapman-Kolmogorov-Gleichung} bekannt.\\

Um im Folgenden auf die Theorie der Dirichlet Formen zurückgreifen zu können, definiere für $ f \in L^2(D) $:
\begin{equation}
\label{eq:Halbgruppe-Definition}
P_t f(x) := \int_D p(t,x,y) f (y) dy.
\end{equation}
Dies definiert eine stark-stetige, kontraktive $ L^2(D) $-Halbgruppe $ (P_t)_{t>0} $. Ferner definiere für $ (P_t)_{t>0} $ den infinitesimalen Generator.

\begin{defin} Der \emph{infinitesimale Generator} $ A: \mathcal{D}[A] \subset L^2(D) \to L^2(D) $ einer stark-stetigen Halbgruppe $ (P_t)_{t>0} $ auf $ L^2(D) $ ist der Operator 
\begin{equation}
\label{eq: Generatordef}
Au := \lim_{t \to 0} \frac{1}{t} \left( P_tu - u  \right),
\end{equation}
definert für jedes $ u $ in dem Definitionsbereich
\begin{equation*}
\mathcal{D}[A] := \left\lbrace u \in L^2(D) : \lim_{t \to 0} \frac{1}{t} \left( P_tu - u  \right) < \infty \right\rbrace.
\end{equation*}
\end{defin}

Man kann zeigen, dass der infinitesimale Generator $ A $ einer stark-stetigen Halbgruppe ein abgeschlossener und dicht definierter linearer Operator auf $ L^2(D) $ ist, der die Halbgruppe eindeutig festlegt (siehe Chap. II, Theorem 1.4. in \cite{Engel-Nagel}). Ferner ist $ A $ ein negativ semi-definiter, selbstadjungierter Operator (siehe Lemma 1.3.1, \cite{Fukushima})

\begin{bem}
Im Folgenden bezeichne $ \Delta / 2 $ oder oft einfach nur $ \Delta $ sowohl den infinitesimalen Generator der stark-stetigen Halbgruppe $ P_t $ als auch den Laplace-Operator; dabei sollte es vom Kontext her klar sein, ob nun der Generator oder der Laplace gemeint ist. Als Generator einer Halbgruppe ist $ \Delta $ mehr als nur der Laplace-Operator. Jede Funktion $ u \in \mathcal{D}[\Delta] $ erfüllt eine laterale Randbedingung. Falls $ D $ einen $ C^3 $-Rand hat, ist diese Bedingung einfach $ \partial u / \partial n  = 0 $ auf dem Rand (siehe Example 1.3.2, \cite{Fukushima}).
\end{bem}

Des Weiteren betrachte die $ \lambda $-Resolvente $ G_{\lambda} := ( \lambda - \Delta / 2 )^{-1} : L^2(D) \to \mathcal{D}[\Delta] $ für ein $ \lambda > 0 $. Aufgrund der Kontraktivität der Halbgruppe $ P_t $ erhält man für $ \lambda > 0 $ und $ u \in \mathcal{D}[\Delta] $ die äquivalente Charakterisierung:
\begin{equation}
\label{eq:Reslovente-Definition}
G_{\lambda} u = \int_0^{\infty} \exp( - \lambda t ) P_t u dt,
\end{equation}
mit dem glatten $ \lambda $-Resolventenkern $ G_{\lambda} (x,y) = \int_0^{\infty} \exp( - \lambda t ) p(t,x,y) dt $, der außerhalb der Diagonalen auf $ D \times D $ glatt ist.

Ferner erhält man die zum infinitesimalen Generator $ \Delta / 2 $ assoziierte Dirichlet Form $ \mathcal{E} : \mathcal{D}[\mathcal{E}] \times \mathcal{D}[\mathcal{E}] \to \mathbb{R} $, die durch

\begin{align}
\label{eq:DirichletForm-Definition}
\begin{split}
& \mathcal{D}[\mathcal{E}] = \mathcal{D}[\sqrt{- \Delta}] = H^1(D), \\
& \mathcal{E}(u,v) = \frac{1}{2} \left( \sqrt{- \Delta} u, \sqrt{- \Delta} v \right)_{L^2(D)} = \frac{1}{2} \int_D \nabla u(x) \nabla v(x) dx
\end{split}
\end{align}
definiert ist (siehe Theorem 1.3.1 und Theorem 1.4.1 in \cite{Fukushima}). Dabei ist
\begin{equation}
\label{eq:H^1(D)-Norm-Definition}
H^1(D) := \left\lbrace u \in L^2(D) : \| u \|_{H^1(D)}^2 := (u,u)_{L^2(D)} + \mathcal{E}(u,u) < \infty  \right\rbrace .
\end{equation}

\begin{bem}
Es gilt die Inklusion: $ \mathcal{D}[\Delta] \subset \mathcal{D}[\mathcal{E}] = H^1(D) \subset L^2(D) $.
\end{bem}



\subsection{Vorüberlegungen}


Um die reflektierte Brownsche Bewegung auf $ \overline{D} $ zu konstruieren, muss der Übergangskern $ p(t,x,y) $ und damit auch die Halbgruppe $ P_t $ auf $ \overline{D} $ stetig erweitert werden. Dabei bezeichnet $ \overline{D} $ den euklidischen Abschluss von $ D $ in $ \mathbb{R}^d $.

\begin{lemma}
Sei $ u \in C^2(D) $,  $u$ und $\Delta u$ beschränkt in $D$, und $ u \in \mathcal{D}[\Delta] $. Dann ist $u$ gleichmäßig stetig auf $D$ und kann zu einer stetigen Funktion auf $ \overline{D} $ erweitert werden.
\end{lemma}

Der Beweis erfolgt mittels Approximation des Lipschitz Gebietes von Innen durch glatt berandete Gebiete, und der Eigenschaften der $ \lambda $-Resolvente. Für weitere Details siehe Lemma 4.2 in \cite{Bass-Hsu}. Mithilfe dieses Lemmas gilt es nun den Übergangskern und den Resolventenkern stetig auf $ \overline{D} $ zu erweitern. 

\begin{lemma}
\label{sec:lemma2}
\begin{enumerate}
\item[(i)] Der Übergangskern $ p(t,x,y) $ definiert auf $ (0, \infty ) \times D \times D $ kann stetig zu einer Funktion auf $ (0, \infty ) \times \overline{D} \times \overline{D} $ erweitert werden, so dass er zu einem Übergangskern auf $ \overline{D} $ wird. Weiterhin gilt: $ p(t,x,y) \in \mathcal{D}[\Delta] $.
\item[(ii)] Der Resolventenkern $ G_{\lambda}(x,y) $, definiert auf $ D \times D $ außerhalb der Diagonale, kann stetig auf $ \overline{D} \times \overline{D} $ außerhalb der Diagonalen erweitert werden, so dass er ein Resolventenkern auf $ \overline{D} $ wird.
\item[(iii)] Für alle $ (x,y) \in \overline{D} \times \overline{D} $ und $ \lambda > 0 $ gilt:
\begin{equation}
\label{eq:lemma2-G}
G_{\lambda}(x,y) = \int_0^{\infty}  \exp(-\lambda t) p(t,x,y) dt.
\end{equation}
\end{enumerate}
\end{lemma}

\begin{proof}
Um zu zeigen, dass $ p(t,x,y) $ stetig auf $ \overline{D} \times \overline{D} $ erweitert werden kann, wird das vorangegangene Lemma sowohl auf $ p(t,x,y) $ als Funktion von $ x \mapsto p(t,x,y) $ als auch als Funktion von $ y \mapsto p(t,x,y) $ angewendet. Die wichtigsten Eigenschaften übertragen sich auf die Erweiterung mithilfe der Grenzwertbildung und der dominierten Konvergenz.\\
 
Betrachte zunächst $ p(t,x,y) $ auf $ D $. Da der Übergangskern $ p(t,x,y) $ glatt auf $ (0,\infty) \times D \times D $ angenommen werden kann, folgt, dass $ p(t, \cdot , y) \in C^2(D) $ und $ p(t, \cdot, y) $ beschränkt durch $ 1 $ auf $ D $ ist. Weiterhin  ist der Laplace des Übergangskerns gleichmäßig beschränkt in $ (x,y) \in D $:
\begin{equation} 
| \Delta p(t,x,y) |  \leq C \sup_{z \in D} p(t/2, z,z)
\end{equation}
für eine positive Konstante $ C $ (siehe \cite{Bass-Hsu}, Gleichung (2.10)). Somit erfüllt $ p(t, \cdot , y) $ alle Bedingungen des vorangegangenen Lemmas bis auf $ p(t, \cdot, y) \in \mathcal{D}[\Delta] $.

Betrachte dazu die zu $ p(t, \cdot , \cdot ) $ assoziierte $ L^2 $-Halbgruppe $ P_t $, wie in (\ref{eq:Halbgruppe-Definition}) definiert. Dann kann man die Funktion $ p(t,x,y) $ wie folgt mithilfe der zu $ -\Delta / 2$ gehörigen 1-Resolvente $ G_1 = (1 - \Delta / 2)^{-1} :L^2(D) \to \mathcal{D}[\Delta] $ darstellen:
\begin{equation}
p(t, \cdot , y) = G_1 G_1 f(\cdot),
\end{equation}
für $ f(x) := (1 - \Delta_x / 2)^2 p(t,x,y) $, dabei bezeichnet der Index $ x $, dass der Laplace auf die $ x $ Variable angewendet wird.
Betrachte die Funktion $ f $ genauer:
\begin{align*}
f(x) & = (1 - \Delta_x / 2)^2 p(t,x,y) \\
& = p(t,x,y) - \Delta_x p(t,x,y) + \frac{1}{4} (\Delta_x P_{t/2})(\Delta_x p(t/2,x,y)), 
\end{align*}
wobei man den letzen Term der rechten Seite mithilfe der Chapman-Kolmogorov-Gleichung erhält: 
\begin{align*}
\Delta_x \left( \Delta_x p(t,x,y) \right) & =  \Delta_x \left( \Delta_x \left( \int_D p(t/2, x,z) p(t/2,z,y)dz \right) \right) \\
& = (\Delta_x P_{t/2})(\Delta_x p(t/2,x,y)). 
\end{align*}
Also ist $ f $ als Summe beschränkter Funktionen beschränkt auf $ D $ und somit ist $ G_1 f $ im Abschluss von $ \mathcal{D}[\Delta] $ und $ p = G_1 G_1 f $ in $ \mathcal{D}[\Delta] $. 
Somit folgt mithilfe des vorangegangenen Lemmas, dass $ x \mapsto p(t, x , y ) $ zu einer stetigen Funktion auf $ \overline{D} $ erweitert werden kann. Aufgrund der Symmetrie von $ p(t,x,y) $ in $ (x,y) $ gilt dies analog für $ y \mapsto p(t, x , y ) $.

Nun bleibt zu zeigen, dass $ p(t,x,y) $ stetig auf $ \overline{D} \times \overline{D} $ ist. Sei $ (x,y) $ in $ \overline{D} \times \overline{D} $ und $ ((x_n, y_n))_{n \in \mathbb{N}} \subseteq D \times D $ eine gegen $ (x,y) $ konvergente Folge. Die Konservativität und die Chapman-Kolmogorov Gleichung übertragen sich durch die gleichmäßige Beschränktheit von $ p(t,x,y) $ und dominierter Konvergenz auf die Erweiterung des Übergangskerns. 
Für die Konservativität von $ p(t,x,y) $ auf $ \overline{D} \times \overline{D} $ gilt:
\begin{align*}
\int_{D} p(t,x,y) dy & = \int_{D} \lim_{n \to \infty} p(t,x_n,y_n) dy  = \lim_{n \to \infty}  \int_{D} p(t,x_n,y_n) dy  = 1,
\end{align*}
da $ p(t,x,y) $ konservativ auf $ D \times D $ ist. Weiterhin gilt die Chapman-Kolmogorov Gleichung auf $ \overline{D} \times \overline{D} $:
\begin{align*}
p(s+t,x,y) & = \lim_{n \to \infty}  p(s+t,x_n,y_n)  = \lim_{n \to \infty}  \int_{D} p(s,x_n,z) p(t,z,y_n) dz \\
& = \int_{D} \lim_{n \to \infty} p(s,x_n,z) p(t,z,y_n) dz  = \int_{D} p(s,x,z) p(t,z,y) dz. 
\end{align*}
Damit kann nun die gemeinsame Stetigkeit von $ p(t,x,y) $ gezeigt werden. Es existiert aufgrund der Beschränktheit von $ p(t,x,y) $ eine Majorante, so dass mit dominierter Konvergenz und der Chapman-Kolmogorov Gleichung folgt:
\begin{align*}
& \lim_{n \to \infty} p(t,x,y) - p(t,x_n,y_n) \\
& = \lim_{n \to \infty} \int_{D} p(t/2,x,z) p(t/2,z,y) dz - \int_{D} p(t/2,x_n,z) p(t/2,z,y_n) dz \\
& = \int_{D} \lim_{n \to \infty} \left[ p(t/2,x,z) p(t/2,z,y) - p(t/2,x_n,z) p(t/2,z,y_n) \right] dz = 0, 
\end{align*}
wegen der komponentenweise Stetigkeit in $x$ und $y$.

Somit wäre (i) gezeigt.
\newline

Den Resolventenkern $ G_{\lambda}(x,y) $ kann man nun auch auf $ \overline{D} \times \overline{D} $ stetig erweitern, und die obige Darstellung (\ref{eq:lemma2-G}) gilt auch für die Erweiterung. Sei $ (x,y) $ in $ \overline{D} \times \overline{D} $ und $ ((x_n, y_n))_{n \in \mathbb{N}} \subseteq D \times D $ eine gegen $ (x,y) $ konvergente Folge, dann folgt mit dominierter Konvergenz und der Abschätzung $ p(t,x,y) \leq c_1 t^{-d/2} \exp(-|x-y|^2 / c_2t) $ (siehe Theorem 3.1 in \cite{Fukushima}): 
\begin{align*}
G_{\lambda}(x,y) & = \lim_{n \to \infty} G_{\lambda}(x_n,y_n) \\
& = \lim_{n \to \infty} \int_0^{\infty} \exp(- \lambda t ) p(t,x_n,y_n) dt = \int_0^{\infty} \exp(- \lambda t ) p(t,x,y) dt
\end{align*}
Das beweist (ii) und (iii).

\end{proof}




\subsection{Existenz der Reflektierten Brownschen Bewegung}
\label{sec:Thm-RBM}

Nachdem der Übergangskern $ p(t,x,y) $ auf $ \overline{D} $ erweitert worden ist, erfolgt nun die Konstruktion der zugehörigen reflektierten Brownschen Bewegung.

\begin{thm}
\label{sec: Thm-MP}
Es existiert ein eindeutiger $ \overline{D} $-wertiger, stetiger, normaler, starker Markov-Prozess, dessen assoziierte Dirichlet-Form $ \mathscr{E} $ ist.
\end{thm}

\begin{bem}
Dieser so konstruierte Markov-Prozess wird im Folgenden als die \emph{reflektierte Brownsche Bewegung (RBM)} auf $ D $ bezeichnet.
\end{bem}

\begin{proof}
Dieser Beweis gliedert sich in zwei Teile. Zunächst muss nachgewiesen werden, dass die betrachtete Dirichlet Form $\mathcal{E}$ aus (\ref{eq:DirichletForm-Definition}) auf dem beschränkten Lipschitz-Gebiet $D$ regulär ist, um so die Existenz eines assoziierten Hunt Prozesses zu folgern. Im zweiten Teil gilt es, die geforderten Eigenschaften  des so generierten Hunt Prozesses nachzurechnen.

\emph{Teil 1:} Um die allgemeine Theorie der Dirichlet Formen (\cite{Fukushima}, Thm 7.2.1 und Thm 4.5.1) anwenden zu können, müssen wir zeigen, dass die Dirichlet Form $ \mathcal{E} $ regulär ist. Mit anderen Worten ist zu zeigen, dass die Menge $ Z := H^1(D) \cap C(\overline{D}) $, auch Kern von $ \mathcal{E} $ genannt, dicht in $ H^1(D) $ bezüglich der $ H^1 $-Norm ist und dicht in $ C(\overline{D}) $ bezüglich der Supremums-Norm.

Zeige zunächst, dass $ Z $ dicht in  $ C(\overline{D}) $ ist bezüglich der Supremums-Norm. Sei $ u \in C(\overline{D}) $. Finde eine Folge $ (u_n)_{n \in \mathbb{N} } \subseteq Z $, so dass $ u_n \to u $ für $ n \to \infty $ in der Supremums-Norm. Bemerke dazu, dass jede stetige Funktion gleichmäßig auf Kompakta mittels glatten Funktionen approximiert werden kann [Approximationssatz von Weierstraß]. Da $ \overline{D} \subset \mathbb{R}^d $ abgeschlossen und beschränkt ist, somit kompakt ist, wähle $ \overline{D} $ als Kompaktum. Eine solche approximierende Folge von Funktionen ist sicher in $ Z $ enthalten.

Zeige nun, dass $Z$ dicht in $H^1(D)$ bezügliche der $ H^1 $-Norm ist. Sei $ u \in H^1(D) $. Finde eine Folge $ (u_n)_{n \in \mathbb{N} } \subseteq Z $, so dass $ u_n \to u $ für $ n \to \infty $ in der $ H^1 $-Norm. Bemerke dazu, dass aufgrund der besonderen Form der $H^1(D)$-Norm, 
\begin{equation}
\| u \|^2_{H^1(D)} = \mathcal{E} (u,u) + (u,u)_{L^2(D)} = \frac{1}{2} \int_D | \nabla u(x) |^2 dx + \int_D | u(x) |^2 dx,
\end{equation}
die beschränkten Funktionen in $H^1(D)$, im Folgenden mit $ H^1_b(D) $ bezeichnet, dicht in $H^1(D)$ sind. Betrachte dazu $ v \in H^1(D) $ und setze $ v_n := (v \wedge n) $ für $ n \in \mathbb{N} $. Die $v_n$ sind beschränkt und in $ H^1(D) $ für alle $ n \in \mathbb{N} $. Außerdem gilt
\begin{equation}
\mathcal{E}(v_n,v_n) \leq \mathcal{E}(v,v) \, \text{ und } \, \mathcal{E}(v_n,v_n) \to \mathcal{E}(v,v) \quad \text{für } n \to \infty.
\end{equation}
Insgesamt also $ v_n \to v $ in $ H^1(D) $. Somit können wir oBdA annehmen, dass $ u \in H^1_b(D) $. Deshalb definiere folgende wohldefinierte Funktionenfolge: $ u_{\lambda} := \lambda G_{\lambda} u \; \text{ für } \lambda > 0$. Dies ist die gewünschte Folge mit $ u_{\lambda} \to u $ in $ H^1(D) $ für $ \lambda \to \infty $. Es gilt $ u_{\lambda} \in H^1(D) $ und $ u_{\lambda} \to u $ in $ L^2(D) $ mit der allgemeinen Theorie der (stark-stetigen) Halbgruppen.

Andererseits gilt:
\begin{equation}
H^1(D) = \mathcal{D}[\sqrt{-\Delta}] = \mathcal{D}[\mathcal{E}].
\end{equation}
Mithilfe dieser Identifikation der Dirichlet Form und der Symmetrie des $L^2$-Skalarproduktes folgt:
\begin{align*}
\mathcal{E}\left(u_{\lambda} - u, u_{\lambda} - u\right) & = \left( \sqrt{-\Delta} \left(u_{\lambda} - u\right), \sqrt{-\Delta} \left(u_{\lambda} - u\right) \right)_{L^2(D)} \\
& = \left( \sqrt{-\Delta} u, \sqrt{-\Delta}  u \right)_{L^2(D)}  + \left( \sqrt{-\Delta} u_{\lambda}, \sqrt{-\Delta} u_{\lambda} \right)_{L^2(D)} \\
& \quad -2 \left( \sqrt{-\Delta} u_{\lambda}, \sqrt{-\Delta} u \right)_{L^2(D)}.
\end{align*}
Nutzt man die Definition von $ u_{\lambda} := \lambda G_{\lambda} u $ und die Eigenschaft, dass $ \sqrt{-\Delta} G_{\lambda} u = G_{\lambda} \sqrt{-\Delta} u $ in $L^2$ für $ u \in H^1_b(D) $ gilt, so erhält man:
\begin{align*}
\mathcal{E}\left(u_{\lambda} - u, u_{\lambda} - u\right) & = \left( \sqrt{-\Delta} u, \sqrt{-\Delta}  u \right)_{L^2(D)}  + \lambda^2 \left( G_{\lambda} \sqrt{-\Delta} u, G_{\lambda} \sqrt{-\Delta} u \right)_{L^2(D)} \\
& \quad -2 \lambda \left( G_{\lambda} \sqrt{-\Delta} u, \sqrt{-\Delta} u \right)_{L^2(D)} \\
& \to \left( \sqrt{-\Delta} u, \sqrt{-\Delta}  u \right)_{L^2(D)}  + \left( \sqrt{-\Delta} u, \sqrt{-\Delta} u \right)_{L^2(D)} \\
& \qquad -2 \left( \sqrt{-\Delta} u, \sqrt{-\Delta} u \right)_{L^2(D)}, \qquad \text{ für } \lambda \to \infty, 
\end{align*}
wobei ausgenutzt wurde, dass laut allgemeiner Halbgruppentheorie $ \lim_{\lambda \to \infty} \lambda G_{\lambda} u = u $ in $L^2$. Insgesamt erhält man $ u_{\lambda} \to u $ in $H^1(D)$. Es bleibt noch zu zeigen, dass $ u_{\lambda} \in C(\overline{D}) $. Es reicht dazu zu zeigen, dass die Resolvente $ G_{\lambda} $ Funktionen aus $H^1_b(D)$ nach $ C(\overline{D}) $ abbildet. Sei $u \in H^1_b(D)$, mithilfe von Lemma~\ref{sec:lemma2} ist $ P_s u $ stetig auf $ \overline{D} $ für alle $ s > 0 $, denn für $ x \in \overline{D} $:
\begin{align*}
\lim_{x_n \to x} P_s u(x) - P_s u(x_n) & = \lim_{x_n \to x} \int_D u(y) (p(s,x,y) - p(s,x_n,y)) dy \\
& =\int_D u(y) \lim_{x_n \to x} (p(s,x,y) - p(s,x_n,y)) dy = 0,
\end{align*}
mit dominierter Konvergenz und da $p(s,x,y)$ stetig auf $ \overline{D} \times \overline{D} $. Damit ist auch $ G_{\lambda}u $ stetig auf $ \overline{D} $, denn mittels der Stetigkeit von $ P_s u $ auf $ \overline{D} $ und dominierter Konvergenz folgt für $ x \in \overline{D} $:
\begin{align*}
\lim_{x_n \to x} G_{\lambda} u(x) - G_{\lambda} u(x_n) & = \lim_{x_n \to x} \int_0^{\infty} \exp(- \lambda t ) (P_t u(x) - P_t u(x_n)) dt \\
& = \int_0^{\infty} \exp(- \lambda t )  \lim_{x_n \to x} (P_t u(x) - P_t u(x_n)) dt = 0.
\end{align*}
Damit ist Teil 1 bewiesen.

\emph{Teil 2:} Bisher ist gezeigt, dass $ \mathcal{E} $ auf einem beschränkten Lipschitz-Gebiet eine reguläre Dirichlet Form auf $L^2(D)$ ist. Dies impliziert mithilfe der allgemeinen Theorie der Dirichlet Formen, dass ein Hunt Prozess $(X_t, P^x ) $ für $ (t,x) \in \mathbb{R}_+ \times \overline{D} $ assoziiert zu $ \mathcal{E} $ exisitiert (\cite{Fukushima}, Theorem 7.2.1). Dies ist die gewünschte reflektierte Brownsche Bewegung auf $ \overline{D} $, die a priori aber nicht alle geforderten Bedingungen erfüllt. Ein solcher Prozess ist im allgemeinen nicht stetig und nur für quasi-jeden Startpunkt $ x \in \overline{D} $  definiert, also auf $ \overline{D} \setminus N $, wobei $N \subseteq \overline{D}$ eine Teilmenge von Kapazität Null ist. Es gilt zunächst zu zeigen, dass dieser Hunt Prozess konsistent auf jeden Startpunkt $ x \in \overline{D} $ erweitert werden kann; die Stetigkeit der Pfade folgt, wenn die Dirichlet Form $ \mathcal{E} $ die lokale Eigenschaft erfüllt.

Um die Existenz des Hunt Prozesses, insbesondere die Existenz der Verteilungen $P^x$ auf $ \overline{D} $ für jedes $ x \in \overline{D} $ zu beweisen, nutze die Tatsache, dass die Verteilungen $P^x$ durch die Existenz des Hunt Porzesses bereits bis auf eine Menge $N$ von Kapazität Null definiert ist. Sei  $ \left\lbrace C([0, \infty); \overline{D}), F ,F_t \right\rbrace $ der Standardpfadraum. Definiere nun für jedes $ x \in \overline{D} $ die Verteilung $ P^x $ durch 
\begin{equation}
\label{egn:Verteilung}
\forall \, t >0 ,\, A \in F: \qquad P^x [A \circ \theta_t ] := \int_D p(t,x,y) P^y [A] dy,
\end{equation}
wobei $ \theta_t : \Omega \to \Omega $ der Shift-Operator $ \theta_t(\omega) = (s \mapsto \omega (s + t)) $ und $ F $ ist die durch $ X_t $ erzeugte Filtration. Der Ausdruck auf der rechten Seite von Gleichung~(\ref{egn:Verteilung}) ist wohldefiniert, da $N \subset \overline{D} $ eine Menge von Maß Null ist. Damit erhält man eine Familie von Wahrscheinlichkeitsmaßen $ \{P^x \}_{x \in \overline{D}} $ auf $ F $. Man kann mit größerem Aufwand zeigen, dass $ (X_t, P^x)_{(t,x) \in \mathbb{R}_+ \times \overline{D} } $ ein zu der Dirichlet Form $ \mathcal{E} $ assoziierter normaler, starker Markov-Prozess ist.

Um zu zeigen, dass $X$ stetige Pfade hat, das heißt
\begin{equation}
P^x[X_t \text{ ist stetig in } t \in [0, \infty)] = 1, \qquad \text{für jedes } x \in D,
\end{equation}
reicht es aufgrund der allgemeinen Theorie der Dirichlet Formen (Theorem 4.5.1, \cite{Fukushima}) zu zeigen, dass $ \mathcal{E} $ die lokale Eigenschaft erfüllt: Falls $ u,v \in H^1(D) $ und die Maße $ \mu := u(x)dx, \; \nu := v(x)dx $ disjunkten Träger haben, dann gilt $ \mathcal{E}(u,v) = 0 $. 
Dies folgt mithilfe folgender Aussage: Sei $ w \in W^1(D) $. Dann ist $ Dw = 0 $ fast überall auf jeder Menge, auf der $w$ konstant ist (siehe Lemma 7.7, \cite{Gilbarg-Trudinger}). Denn damit kann man für $ u,v \in H^1(D) $, wie oben, das Integral aufteilen:
\begin{align*}
\mathcal{E}(u,v) & = \int_D \nabla u(x) \nabla v(x) dx \\
& = \int_{D_{\mu}} \nabla u(x) \nabla v(x) dx + \int_{D_{\nu}} \nabla u(x) \nabla v(x) dx \-- \int_{D_{\mu , \nu}} \nabla u(x) \nabla v(x) dx,
\end{align*}
mit $ D_{\mu} := D \setminus supp (\mu) $, $ D_{\nu} := D \setminus supp (\nu) $ und $ D_{\mu , \nu} := D \setminus ( supp (\mu) \cup supp (\nu) ) $. Auf jedem dieser Integrationsgebiete verschwindet $ \nabla u $ oder $ \nabla v $ fast überall, somit verschwindet jedes der drei Integrale. Also besitzt $ \mathcal{E} $ die lokale Eigenschaft und die Trajektorien von $X$ sind somit stetig.

\end{proof}



\subsection{Existenz der Lokalzeit}
\label{sec:Thm_Lokalzeit}

Nachdem die Existenz der reflektierten Brownschen Bewegung in $ \overline{D} $ bewiesen ist, folgt nun die zugehörige Lokalzeit $L$ der RBM $X$ auf dem Rand. Dabei ist die Lokalzeit $L$ ein positives, stetiges, additives Funktional zu dem Hunt Prozess $X$ (siehe \cite{Fukushima} für eine Einführung). Bemerke, dass $ \sigma $ das Oberflächenmaß des Randes bezeichnet. 

\begin{thm}
\label{sec:Thm-Lokalzeit}
Es existiert ein eindeutiges positives, stetiges additives Funktional $ L_t $ (die Lokalzeit des Randes), so dass für jedes $ x \in \overline{D} $ und jedes $ \lambda > 0 $:
\begin{equation}
\label{eq:Thm-Lokalzeit-behauptung}
G_{\lambda}\sigma(x) = E^x \left[ \int_0^{\infty} \exp(- \lambda t) dL_t \right].
\end{equation}
\end{thm}

\begin{bem}
Die Funktion $ G_{\lambda}\sigma \in \mathcal{D}[\mathcal{E}] $ heißt auch $ \lambda $-Potential des Radon-Maßes $ \sigma $ bezüglich der Dirichlet Form $ \mathcal{E} $.
\end{bem}

\begin{proof}
Wie beim Beweis von Theorem~\ref{sec: Thm-MP} gliedert sich der Beweis in zwei Teile. Im ersten Teil wird nachgewiesen, dass das Oberflächenmaß $ \sigma $ die Voraussetzung von Theorem 5.1.1 aus \cite{Fukushima} erfüllt, das bedeutet, dass das Maß $ \sigma $ ein endliches Energie-Integral hat. Im zweiten Teil dieses Beweises, muss die obige Aussage, die in der allgemeinen Theorie der Dirichlet Formen nur für quasi-alle $ x \in \overline{D} $ gilt, auf alle $ x \in \overline{D} $ erweitert werden.

\emph{Teil 1:} Zeige dazu, dass das Oberflächenmaß des Randes $ \sigma $ ein endliches Energie-Integral hat, das heißt,
\begin{equation}
\label{eq:Integral-sigma-def}
\forall \, u \in H^1(D) \cap C(\overline{D}): \int_D u(x) \sigma(dx) =: \left\langle u, \sigma \right\rangle \leq C \sqrt{\| u \|_{H^1(D)}},
\end{equation} 
für eine positive Konstante $C$.

Um dies zu zeigen, behaupte zunächst: $ G_{\lambda} \sigma \in \mathcal{D}[\sqrt{-\Delta}] = H^1(D) $, wobei 
\begin{equation}
G_{\lambda} \sigma (x) = \int_{\partial D} G_{\lambda} (x,y) \sigma (dy)
\end{equation}
und $ G_{\lambda} (x,y) $ der bekannte Resolventenkern ist. Sei 
\begin{equation}
D_{\epsilon} := \{ x \in \overline{D} : dist(x, \partial D) \leq \epsilon \} \text{ und } \sigma_{\epsilon} (dx) := \frac{1}{\epsilon} \mathsf{1}_{D_{\epsilon}} (x) dx.
\end{equation}

Bemerke dazu, dass aufgund der $ L^2 $-Konvergenz von $ \sigma_{\epsilon} $ zu $ \sigma $ für $ \epsilon \to 0 $, sich das Integral bezüglich $ \sigma $ in (\ref{eq:Integral-sigma-def}) als Limes des $ L^2 $-Skalarproduktes schreiben lässt:
\begin{equation*}
\left\langle u, \sigma \right\rangle = \lim_{\epsilon \to 0 } \frac{1}{\epsilon} \left( u,  \mathsf{1}_{D_{\epsilon}} \right)_{L^2(D)}.
\end{equation*}

Die Tatsache, dass $ G_{\lambda} (x,y) $ stetig auf $ \overline{D} \times \overline{D} $ außerhalb der Diagonalen ist (siehe Lemma~\ref{sec:lemma2}) und die Abschätzung (Gleichung (3.10), \cite{Fukushima}): $ G_{\lambda} (x,y) \leq C | x - y |^{-d+2} $ impliziert, dass $ G_{\lambda} \sigma_{\epsilon} (x) $ gleichmäßig beschränkt ist in $ (x, \epsilon ) $ und dass $ G_{\lambda} \sigma_{\epsilon} (x) $ gleichmäßig gegen $ G_{\lambda} \sigma (x) $ auf $ \overline{D} $ konvergiert für $ \epsilon \to 0 $. Aus $ G_{\lambda} \sigma_{\epsilon} (x) \to G_{\lambda} \sigma (x) $ gleichmäßig auf $ \overline{D} $ für $ \epsilon \to 0 $ folgt $ G_{\lambda} \sigma_{\epsilon} \to G_{\lambda} \sigma  $ in $ L^2(D) $ für $ \epsilon \to 0 $.

Die so definierte Folge $ G_{\lambda} \sigma_{\epsilon} \to G_{\lambda} \sigma  $ konvergiert sogar in $ H^1(D) $. Bemerke zunächst, dass für alle $ \epsilon > 0 $ gilt: $ G_{\lambda} \sigma_{\epsilon} \in H^1(D) $, denn mithilfe der definierenden Eigenschaft des $ \lambda $-Potentials erhält man (siehe dazu Kapitel 2.2, \cite{Fukushima}):
\begin{align*}
\| G_{\lambda} \sigma_{\epsilon} \|^2_{H^1(D)} & = \mathcal{E}_{\lambda} \left(G_{\lambda} \sigma_{\epsilon}, G_{\lambda} \sigma_{\epsilon}\right) - \lambda \left( G_{\lambda} \sigma_{\epsilon}, G_{\lambda} \sigma_{\epsilon}\right)_{L^2(D)} \\
& = \int_{D_{\epsilon}} G_{\lambda} \sigma_{\epsilon} (x) \sigma_{\epsilon} (dx) - \lambda \int_D ( G_{\lambda} \sigma_{\epsilon} (x) )^2 dx < \infty,
\end{align*}
da $ G_{\lambda} \sigma_{\epsilon}, \, D \text{ und } \, D_{\epsilon} $ beschränkt sind. Außerdem erhält man mit der Selbstadjungiertheit des Generators $ - \Delta / 2 $, der Darstellung der $ \mathcal{E}_{\lambda} $-Norm und der definierenden Eigenschaft des $ \lambda $-Potentials $ G_{\lambda} \sigma_{\epsilon} $ für $ \epsilon , \delta \to 0 $:
\begin{align*}
\left(\sqrt{- \Delta} G_{\lambda} \sigma_{\epsilon}, \sqrt{- \Delta} G_{\lambda} \sigma_{\delta}\right)_{L^2(D)} & = \left(- \Delta G_{\lambda} \sigma_{\epsilon}, G_{\lambda} \sigma_{\delta}\right)_{L^2(D)} \\
& = 2 \left\langle \sigma_{\epsilon} , G_{\lambda} \sigma_{\delta} \right\rangle - 2 \lambda \left( G_{\lambda} \sigma_{\epsilon}, G_{\lambda} \sigma_{\delta}\right)_{L^2(D)} \\
& \to 2 \left\langle \sigma , G_{\lambda} \sigma \right\rangle - 2 \lambda \left( G_{\lambda} \sigma, G_{\lambda} \sigma\right)_{L^2(D)},
\end{align*}
für $ \epsilon, \delta \to 0 $, da $  G_{\lambda} \sigma_{\epsilon} $ in $ L^2 $ konvergiert.

Damit folgt
\begin{align*}
\| \sqrt{- \Delta} G_{\lambda} \sigma_{\epsilon} - \sqrt{- \Delta} G_{\lambda} \sigma_{\delta} \|^2_{L^2(D)} & = (\sqrt{- \Delta} (G_{\lambda} \sigma_{\epsilon} - G_{\lambda} \sigma_{\delta}), \sqrt{- \Delta} (G_{\lambda} \sigma_{\epsilon} - G_{\lambda} \sigma_{\delta}))_{L^2(D)} \\
& = (\sqrt{- \Delta} G_{\lambda} \sigma_{\epsilon}, \sqrt{- \Delta} G_{\lambda} \sigma_{\epsilon})_{L^2(D)} \\
& \quad - 2 (\sqrt{- \Delta} G_{\lambda} \sigma_{\epsilon}, \sqrt{- \Delta} G_{\lambda} \sigma_{\delta})_{L^2(D)} \\
& \quad + (\sqrt{- \Delta} G_{\lambda} \sigma_{\delta}, \sqrt{- \Delta} G_{\lambda} \sigma_{\delta})_{L^2(D)} \\
& \to 0 \qquad \text{ für } \, \epsilon, \delta \to 0.
\end{align*}

Insgesamt folgt damit, dass $  G_{\lambda} \sigma_{\epsilon} $ eine Cauchy-Folge in $ \mathcal{D}[\sqrt{-\Delta}] = H^1(D) $ bildet. Da der Operator $ \sqrt{- \Delta} $ abgeschlossen in $L^2(D)$ ist, aufgrund der Definition der schwachen Ableitungen und der Vollständigkeit von $L^2(D)$ (siehe Example 1.2.3, \cite{Fukushima}), gilt $ G_{\lambda} \sigma \in \mathcal{D}[\sqrt{-\Delta}] = H^1(D) $.


Mit Theorem 5.1.1 in \cite{Fukushima} existiert die Lokalzeit des Randes $ L $ von $ X $ und die Behauptung (\ref{eq:Thm-Lokalzeit-behauptung}) hält quasi überall, also mit Ausnahme einer Menge mit Kapazität Null. Nun gilt es zu zeigen, dass die Behauptung für alle $ x \in \overline{D} $ gilt. Betrachte dazu
\begin{equation}
\label{eq:Thm-Lokalzeit-additivesFunktional}
\begin{aligned}
& \lim_{s \to 0} \left[  \int_0^{\infty} e^{-\lambda t} d(L_{t+s} -L_t)  \right] \\
& \qquad = \lim_{s \to 0} \left[  \int_0^{N} e^{-\lambda t} d(L_{t+s} -L_t) + \int_N^{\infty} e^{-\lambda t} d(L_{t+s} -L_t) \right],
\end{aligned}
\end{equation}
für $ N>0 $ und zeige, dass das Integral für $ s \to 0 $ bezüglich $ L_{t+s} $ mit dem Integral bezüglich $ L_t $ übereinstimmt.
Für den ersten Term der rechten Seite erhält man für festes $ N>0 $, da die Abbildung $ t \mapsto L_t $ stetig ist:
\begin{equation*}
\begin{aligned}
\int_0^{N} e^{-\lambda t} d(L_{t+s} -L_t) & \leq L_{N+s} -L_N - L_s  \to  0, \quad \text{ für }  s \to 0 .
\end{aligned}
\end{equation*}
Und für den zweiten Term der rechten Seite von (\ref{eq:Thm-Lokalzeit-additivesFunktional}) ergibt sich mittels Variablentransformation:
\begin{equation*}
\begin{aligned}
\lim_{N \to \infty} \sup_{s \leq 1} \left[ \int_N^{\infty} e^{-\lambda t} d(L_{t+s} -L_t) \right] & = \lim_{N \to \infty} \sup_{s \leq 1} \left[ \int_{N-s}^{\infty} e^{-\lambda (t+s)} dL_{t} - \int_{N}^{\infty} e^{-\lambda t} dL_{t} \right] \\
& \leq  \lim_{N \to \infty}  \left[ \int_{N-1}^{\infty} e^{-\lambda t} dL_{t} - \int_{N}^{\infty} e^{-\lambda t} dL_{t} \right] \\
& = \lim_{N \to \infty}  \left[ \int_{N-1}^{N} e^{-\lambda t} dL_{t} \right] \\
& = 0.
\end{aligned}
\end{equation*}
Insgesamt erhält man in (\ref{eq:Thm-Lokalzeit-additivesFunktional}), dass der Limes verschwindet.

Da (\ref{eq:Thm-Lokalzeit-behauptung}) fast überall gilt und mithilfe von (\ref{eq:Thm-Lokalzeit-additivesFunktional}) , erhält man für $ x \in \overline{D} $:
\begin{align*}
E^x \left[ \int_0^{\infty} \exp(-\lambda t) dL_t  \right] & = \lim_{s \to 0} E^x \left[ \int_s^{\infty} \exp(-\lambda t) dL_t  \right] \\
& = \lim_{s \to 0} \int_D E^y \left[ \int_0^{\infty} \exp(-\lambda (t+s)) dL_{t+s}  \right] p(s,x,y) dy \\
& = \lim_{s \to 0} e^{-\lambda s} P_s \left( E^x \left[ \int_0^{\infty} \exp(-\lambda t) dL_{t+s}  \right]  \right) \\
& = \lim_{s \to 0} e^{-\lambda s} P_s \left( E^x \left[ \int_0^{\infty} \exp(-\lambda t) dL_t  \right]  \right) \\
& = \lim_{s \to 0} e^{-\lambda s} P_s \left( G_{\lambda} \sigma (x) \right) \\
& = G_{\lambda} \sigma (x).
\end{align*}
Im letzten Schritt ist dabei die Stetigkeit von $ G_{\lambda} \sigma $ auf $ \overline{D} $ genutzt worden.


\end{proof}


In den letzten beiden Abschnitten \ref{sec:Thm-RBM} und \ref{sec:Thm_Lokalzeit} ist gezeigt worden, dass auf beschränkten Lipschitz Gebieten $ D $ ein stetiger starker Markov-Prozess $ X $ und die zugehörigen Lokalzeit $ L $ auf dem Rand für jeden Startpunkt $ x \in \overline{D} $ existieren. Dieser stochastische Prozess $ X $ ist über die Dirichlet Form $ \mathcal{E} $ und der Halbgruppe $ P_t $ mit dem Übergangskern $ p(t,x,y) $ assoziiert, weshalb $ X $ als die reflektierte Brownsche Bewegung bezeichnet wird.

Damit lässt sich nun das Neumann-Randwertproblem probabilistisch lösen.


\subsection{Das Neumann-Randwertproblem für beschränkte Lipschitz Gebiete}
\label{sec:NRWP-section}

Man betrachte das folgende \emph{Neumann-Randwertproblem} für ein beschränktes Lipschitz Gebiet $ D \subseteq \mathbb{R}^d $:
\begin{equation}
\label{eq:NWRP-Definition}
\begin{cases}
\Delta u = 0 & \text{ auf $D$ }\\
\frac{\partial}{\partial n} u = f & \text{ auf $ \partial D $ },
\end{cases}
\end{equation}
wobei $ f \in B( \partial D ) $ (dem Raum der beschränkten Borel-messbaren Funktionen) mit
\begin{equation}
\int_{\partial D} f(x) \sigma (dx) = 0
\end{equation}
und $n$ bezeichnet die äußere Normale auf $ \partial D $.

\begin{defin} 
Eine Funktion $ u \in C(\overline{D}) $ heißt schwache Lösung des Neumann-Randwertproblems (\ref{eq:NWRP-Definition}), falls für alle $ \phi \in C^2( \overline{D}) $ gilt:
\begin{equation}
\label{eq:NRWP-schwache-Lsg}
\int_D u(x) \Delta \phi (x) dx + \int_{\partial D} f(x) \phi(x) \sigma (dx) = \int_{\partial D} u(x) \frac{\partial \phi}{\partial n} (x) \sigma (dx).
\end{equation}
\end{defin}


\begin{thm}
\label{sec:Thm-NRWP}
Sei $D \subseteq \mathbb{R}^d $ ein beschränktes und zusammenhängendes Lipschitz Gebiet und sei $ f \in B( \partial D ) $ mit $ \int_{\partial D} f(x) \sigma (dx) = 0 $. Dann existiert eine eindeutige schwache Lösung $ u \in C(\overline{D}) $ für das Neumann-Randwertproblem (\ref{eq:NWRP-Definition}) mit der Normierung $ \int_D u(x) dx = 0 $. Des Weiteren gilt für $ x \in D $ die Darstellung:
\begin{equation}
\label{eq:thm-NRWP-def}
u(x) = \lim_{t \to \infty} \frac{1}{2} E^x \left[ \int_0^t f(X_s) dL_s \right],
\end{equation}
wobei $X$ eine reflektierte Brownsche Bewegung auf $D$ ist und $L_t$ die Lokalzeit von $X$ auf dem Rand $\partial D $ ist.
\end{thm}

Im Beweis des obigen Theorems verwendet man das folgende Lemma, um nachzuweisen, dass die angegebene Funktion $ u $ eine schwache Lösung des Neumann-Randwertproblems ist.  

\begin{lemma} 
Im Setting des obigen Theorems gilt für $ \phi \in C^2( \overline{D})$:
\begin{align}
\begin{split}
\label{eq:NRWP-lemma}
& \frac{1}{2} \int_0^t \left( \int_D p(s,x, \cdot ) \Delta \phi (x) dx \right) ds \\ 
& \qquad \qquad \qquad = P_t \phi - \phi + \frac{1}{2} \int_0^t \left( \int_{\partial D} p(s,x,\cdot )\frac{\partial \phi}{\partial n} (x) \sigma (dx)  \right) ds .
\end{split}
\end{align}
\end{lemma}

In dem Beweis approximiert man das Lipschitz Gebiet von Innen mit glatt berandeten Gebieten und zeigt, dass die Übergangskerne der glatt berandeten Gebiete gegen den Übergangskern des Lipschitz Gebietes konvergieren. Somit genügt es (\ref{eq:NRWP-lemma}) für die glatt berandeten Gebiete zu zeigen, was eine leichte Rechnung mithilfe der zweiten Identität von Gauß-Green ist. Siehe für Einzelheiten Lemma 5.4 in \cite{Bass-Hsu}.

\begin{proof}

Da von D. S. Jerison und C. E. Kenig in \cite{Jerison-Kenig} die Existenz und Eindeutigkeit einer schwachen Lösung des Neumann-Randwertproblems für beschränkte Lipschitz Gebiete bewiesen worden ist, genügt es zu zeigen, dass 
\begin{equation*}
u(x) = \lim_{t \to \infty} \frac{1}{2} E^x \left[ \int_0^t f(X_s) dL_s \right]
\end{equation*}
die Eigenschaft (\ref{eq:NRWP-schwache-Lsg}) einer schwachen Lösung erfüllt.

Dabei gilt für $ t >0 $ und $ x \in D $ für $ f \in B( \partial D ) $ und $ f \geq 0 $:
\begin{align}
\begin{split}
E^x \left[ \int_0^t f(X_s) dL_s  \right] & = \int_0^t \int_{\partial D} p(s,x,y) f(y) \sigma(dy) ds \\
& = \int_0^t \int_{\partial D} \left[ p(s,x,y) -\frac{1}{|D|} \right] f(y) \sigma(dy) ds.
\end{split}
\end{align}
Der erste Schritt beruht auf der Eigenschaft von $ L $ als Lokalzeit von $ X $ auf dem Rand $ \partial D $ und im zweiten Schritt ist ausgenutzt worden, dass das Integral von $ f $ über den Rand verschwindet. Man erhält die Aussagen für beliebiges $ f \in B( \partial D ) $, indem man den Positivteil und Negativteil einzeln betrachtet. Somit folgt für $ 0 < t_1 \leq t_2 < \infty $
\begin{equation*}
\left| E^x \left[ \int_{t_1}^{t_2} f(X_s) dL_s  \right] \right| \leq \int_{t_1}^{t_2} \int_{\partial D} \left| p(s,x,y) -\frac{1}{|D|} \right| |f(y)| \sigma(dy) ds.
\end{equation*}
Mithilfe der Aussage, dass ein $ T > 0 $ und ein $ c>0 $ existiert, so dass 
\begin{equation}
\label{eq:Thm-NRWP-stationäre-Verteilung}
 \left| p(t,x,y) -\frac{1}{|D|} \right| \leq e^{-ct} 
\end{equation} 
für $ t > T  $ und alle $ (x,y) \in \overline{D} \times \overline{D} $ (siehe Theorem 2.4 in \cite{Fukushima}), also der Tatsache, dass der Übergangskern gleichmäßig und exponentiell gegen die stationäre Verteilung der RBM konvergiert, folgert man, dass die Konvergenz in (\ref{eq:thm-NRWP-def}) gleichmäßig ist über $ \overline{D} $ und man erhält die alternative Darstellung:
\begin{equation}
\label{eq:Thm-NRWP-glm-Konvergenz}
u (x) = \frac{1}{2} \int_0^{\infty} \int_{\partial D} \left[ p(s,x,y) -\frac{1}{|D|} \right] f(y) \sigma(dy) ds.
\end{equation}

Aufgrund der Beschränktheit von $ f $, der Stetigkeit von $ p(t,x,y) $ auf $ \overline{D} $ und der oberen Grenze von $ p(t,x,y) $ für $ c_1, c_2, t > 0 $ : $ p(t,x,y) \leq c_1 t^{-d/2} \exp(-|x-y|^2 / c_2t) $ (siehe Theorem 3.1 in \cite{Fukushima}), lässt sich zeigen, dass $ u \in C(\overline{D}) $. Denn die oberen Schranken erlauben es mit dominierter Konvergenz zu rechnen:
\begin{align*}
\lim_{n \to \infty} u(x_n) - u(x) & = \lim_{n \to \infty} \frac{1}{2} \int_0^{\infty} \int_{\partial D} \left[ p(s,x_n,y) - p(s,x,y) \right] f(y) \sigma(dy) ds \\
& = \frac{1}{2} \int_0^{\infty} \int_{\partial D} \lim_{n \to \infty } \left[ p(s,x_n,y) - p(s,x,y) \right] f(y) \sigma(dy) ds = 0 .
\end{align*}
Außerdem rechnet man mit Fubini und der Normiertheit von $ f $: 
\begin{align*}
\int_D u(x) dx & = \frac{1}{2} \int_D \int_0^{\infty} \int_{\partial D} \left[ p(s,x,y) -\frac{1}{|D|} \right] f(y) \sigma(dy) ds dx \\
& = \frac{1}{2} \int_D \int_0^{\infty} \int_{\partial D}  p(s,x,y) f(y) \sigma(dy) ds dx \\
& = \frac{1}{2} \int_0^{\infty} \int_{\partial D} \int_D  p(s,x,y) dx f(y) \sigma(dy) ds \\
& = \frac{1}{2} \int_0^{\infty} \int_{\partial D}  f(y) \sigma(dy) ds = 0.
\end{align*}

Um schließlich zu zeigen, dass $ u $ eine Lösung des schwachen Neumann-Randwertproblems ist, ist (\ref{eq:NRWP-schwache-Lsg}) nachzurechnen. Dazu bezeichne \[ u_t (x) := \frac{1}{2} E^x \left[ \int_0^t f(X_s) dL_s \right] = \frac{1}{2} \int_0^{t} \int_{\partial D} p(s,x,y)  f(y) \sigma(dy) ds, \]wobei $ u_t \to u $ gleichmäßig auf $ D $ und somit auch in $ L^2 $. Für $ \phi \in C^2(\overline{D}) $ erhält man mittels Fubini:
\begin{align*}
\left( u_t , \Delta_x \phi \right)_{L^2(D)} & = \frac{1}{2} \int_D \int_0^{t} \int_{\partial D} p(s,x,y)  f(y) \sigma(dy)\, ds\; \Delta_x \phi (x) \,dx \\
& = \frac{1}{2} \int_{\partial D} f(y) \int_0^{t} \int_D  p(s,x,y) \Delta_x \phi (x) dx \, ds \, \sigma(dy).
\end{align*}
Wendet man nun auf $ \frac{1}{2} \int_0^t \left[ \int_D p(s,x, \cdot ) \Delta_x \phi (x) dx \right] ds $ das vorangegangene Lemma an, so erhält man:
\begin{align}
\begin{split}
\label{eq:Thm-NWRP-rhs}
& \left( u_t , \Delta_x \phi \right)_{L^2(D)} =  \int_{\partial D} f(y) \left[ \frac{1}{2} \int_0^{t}  \int_D  p(s,x,y) \Delta_x \phi (x) dx \, ds \right] \sigma(dy) \\
& \quad = \int_{\partial D} f(y) \left[ P_t \phi(y) - \phi (y) + \frac{1}{2} \int_0^t \left( \int_{\partial D} p(s,x,y )\frac{\partial \phi}{\partial n_x} (x) \sigma (dx)  \right) ds  \right] \sigma(dy) \\
& \quad = \int_{\partial D} f(y) \left[ P_t \phi (y) - \phi (y)  \right] \sigma(dy) \\
& \qquad \qquad + \frac{1}{2} \int_{\partial D} f(y) \int_0^t \int_{\partial D} \left[ p(s,x,y) - \frac{1}{|D|} \right] \sigma(dy) \, ds \frac{\partial \phi}{\partial n_x}(x) \sigma(dx).
\end{split}
\end{align}
Hierbei wurde wieder ausgenutzt, dass $ \int_{\partial D} f(y) \sigma(dy) = 0 $.

Nutze aus, dass $ p(t,x,y) $ gleichmäßig gegen die stationäre Verteilung konvergiert, also
\begin{equation*}
\int_D p(t, \cdot ,y) \phi (y) dy \to \frac{1}{|D|} \int_D \phi (y) dy \quad \text{ gleichmäßig auf $ D $ für } t \to \infty.
\end{equation*}

Betrachte nun den ersten Term der rechten Seite von (\ref{eq:Thm-NWRP-rhs}) für $ t \to \infty $:
\begin{align*}
\int_{\partial D} f(y) \left[ P_t \phi (y) - \phi (y)  \right] \sigma(dy) & = \int_{\partial D} f(y) \left[ P_t \phi (y)\right] \sigma(dy) - \int_{\partial D} f(y) \phi (y) \sigma(dy) \\
& \to \int_{\partial D} f(y) \left[ \frac{1}{|D|} \int_D \phi (x) dx \right] \sigma(dy)  - \int_{\partial D} f(y) \phi (y) \sigma(dy) \\
& = - \int_{\partial D} f(y) \phi (y) \sigma(dy).
\end{align*}
Hierbei wurde wieder ausgenutzt, dass $ \int_{\partial D} f(y) \sigma(dy) = 0 $.

Insgesamt ist $ u $ eine stetige und normierte Funktion auf $ \overline{D} $ und $ u $ erfüllt die geforderte Eigenschaft (\ref{eq:NRWP-schwache-Lsg}) einer schwachen Lösung des Neumann-Randwertproblems. Das beendet den Beweis.

\end{proof}

\subsection{Ausblick}

Das Problem der reflektierten Brownschen Bewegung und damit das Neumann-Randwert-problem lassen sich auf unterschiedlich reguläre Gebiete lösen. Auf Gebieten mit $ C^3 $-Rand erfolgt die Konstruktion pfadweise und kann über das Skorokhod-Problem gelöst werden, wie im ersten Teil gesehen. Diese Methode lässt sich für konvexe Gebiete verallgemeinern, wie von H. Tanaka in \cite{Tanaka} bewiesen, oder auch für beliebige Gebiete, die eine gewisse Randbedingung erfüllen, wie von P. L. Lions und A. S. Sznitman in \cite{Lions-Sznitman} bewiesen. Im Falle eines Gebietes mit Lipschitz-Rand ist eine umfangreichere Theorie von Nöten, was mit einem Verlust der Anschaulichkeit einhergeht. 

Der ideale Rand für das Neumann-Randwertproblem und die reflektierte Brownsche Bewegung, im analytischen und probabilistischen Sinne, ist der Kuramochi-Rand. R. F. Bass und P. Hsu \cite{Bass-Hsu} haben hezeigt, dass im Falle eines beschränkten Lipschitz Gebiets die Kuramochi-Kompaktifizierung mit der Euklidischen Kompaktifizierung übereinstimmt, dass also, wie im zweiten Teil dieser Arbeit gesehen, eine reflektierte Brownsche Bewegung mit stetigen Pfaden auf eben dieser Kompaktifizierung existiert.

\newpage
\appendix

\bibliographystyle{plain}
\bibliography{Literatur}


\newpage
\fancyhead{}
\mbox{}
\newpage

\vspace*{1cm}

\section{Eidesstattliche Erklärung}

Hiermit versichere ich an Eides statt und durch meine Unterschrift, dass die vorliegende Arbeit von mir selbstständig angefertigt worden ist. Inhalte und Passagen, die aus fremden Quellen stammen und direkt oder indirekt übernommen worden sind, wurden als solche kenntlich gemacht. Ferner versichere ich, dass ich keine andere, außer der im Literaturverzeichnis angegebenen Literatur verwendet habe. Die Arbeit wurde bisher keiner Prüfungsbehörde vorgelegt und auch noch nicht veröffentlicht.\\

Bonn, den \today 


\end{document}